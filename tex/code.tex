\section{图论}

\subsection{最短路径}
\subsubsection{Dijkstra}
\begin{lstlisting}
const int maxn = 1e4;
const int inf = 0x3f3f3f3f;

//d数组用来记录源点s到顶点i的最短距离
//v表示该顶点是否在顶点集S中
//g邻接矩阵存图,g[i][j]表示i到j的边的权值,无边时为inf
//n为顶点数量
int d[maxn], v[maxn];
int g[maxn][maxn];
int n;
void dij(int s)
{
	memset(v, 0, sizeof(v));
	for(int i=1;i<=n;i++)
	d[i] = g[s][i];
	v[s] = 1;
	for(int i=1;i<=n;i++)
	{
		int u = 0;
		for(int j=1;j<=n;j++)
		{
			if(!v[j] && (u==0 || d[j] < d[u]))
				u = j;
		}
		if(u==0)return ;
		v[u] = 1;
		for(int j=1;j<=n;j++)
		{
			d[j] = min(d[j], d[u]+g[u][j]);
		}
	}
}
\end{lstlisting}

\subsubsection{Dijkstra优化}
\begin{lstlisting}
const int maxn = 1e4;
const int inf = 0x3f3f3f3f;
typedef pair<int, int> P; //first表示最短距离,second表示顶点编号
//边:to表示这条边指向的顶点,权值为w
struct Edge
{
	int to, w;
};
//用vector实现邻接表
vector<Edge> g[maxn];
int d[maxn]; //记录源点到顶点i的最短距离
int n;

void dij(int s)
{
	priority_queue<P, vector<P>, greater<P> > q;
	memset(d, inf, sizeof(d));
	d[s] = 0;
	q.push(P(0, s));
	while(!q.empty())
	{
		P p = q.top();
		q.pop();
		int u = p.second;
		if(d[u] < p.first) continue;
		for(int i=0; i<g[u].size(); i++)
		{
			Edge e = g[u][i];
			if(d[e.to] > d[u] + e.w)
			{
				d[e.to] = d[u] + e.w;
				q.push(P(d[e.to], e.to));
			}
		}
	}
}
\end{lstlisting}

\subsubsection{Floyd}
\begin{lstlisting}
int g[maxn][maxn];
int n;
void floyd()
{
	for(int k=1;k<=n;k++)
		for(int i=1;i<=n;i++)
			for(int j=1;j<=n;j++)
				g[i][j] = min(g[i][j], g[i][k] + g[k][j]);
}
\end{lstlisting}

\subsubsection{Bellman-Ford}
\begin{lstlisting}
const int maxn = 1e4;
const int inf = 0x3f3f3f3f; //常用于表示无穷大

//边结构体,记录u->v的边,权值为w
struct Edge
{
	int u, v, w;
	Edge(int uu, int vv, int ww) { u=uu; v=vv; w=ww; }
	Edge(){}
}e[maxn];
int edgecnt; // 边的数量
//加边操作
void addEdge(int u, int v, int w)
{
	e[edgecnt++] = Edge(u, v, w);
}

int n; //顶点总数
int d[maxn]; //记录最短距离的数组

//存在负权回路则返回true,否则返回false
bool bellman_ford(int s)
{
	memset(d, inf, sizeof(d));
	d[s] = 0;
	//进行n-1次松弛操作,第n次检查是否含有负权回路
	for(int i=1;i<=n;i++)
	{
		int flag = 0;
		for(int j=0; j<edgecnt; j++)
		{
			Edge t = e[j];
			int u, v, w;
			u = t.u; v = t.v; w = t.w;
			if(d[v] > d[u] + w)
			{
				d[v] = d[u] + w;
				flag = 1;
			}
		}
		if(!flag) return false;
		if(i==n && flag) return true;
	}
	return false;
}
\end{lstlisting}

\subsubsection{SPFA}
\begin{lstlisting}
const int maxn = 1e4;
const int inf = 0x3f3f3f3f; //常用于表示无穷大

//边结构体,to表示边指向的顶点编号,权值为w
struct Edge
{
	int to, w;
	Edge(int tt, int ww) { to = tt; w = ww; }
	Edge(){}
};
//vector实现的邻接表
vector<Edge> g[maxn];
int n;//顶点数
//d表示最短距离, inq[i]表示结点是否在队列中,为1则在,cnt[i]记录i入队的次数
int d[maxn], inq[maxn], cnt[maxn];
//初始化
void init()
{
	memset(d, inf, sizeof(d));
	memset(inq, 0, sizeof(inq));
	memset(cnt, 0, sizeof(cnt));
}
//返回true表示存在负权回路
bool spfa(int s)
{
	init();
	d[s] = 0;
	inq[s] = 1;
	cnt[s] = 1;
	queue<int> q;
	q.push(s);
	while(!q.empty())
	{
		int u = q.front();
		inq[u] = 0;
		q.pop();
		for(int i=0;i < g[u].size(); i++)
		{
			Edge e = g[u][i];
			if(d[e.to] > d[u] + e.w)
			{
				d[e.to] = d[u] + e.w;
				if(inq[e.to] == 0)
				{
					inq[e.to] = 1;
					q.push(e.to);
					cnt[e.to]++;
					if(cnt[e.to] > n) return true;
				}
			}
		}
	}
	return true;
}
\end{lstlisting}


\subsection{LCA}
\subsubsection{倍增}
\begin{lstlisting}
#include<bits/stdc++.h>
using namespace std;
typedef long long ll;
#define inf 0x3f3f3f3f
typedef pair<int, int> P;
const int maxn = 5e5+5;
const ll mod = 1e9+7;

vector<int> son[maxn]; // 存储儿子顶点
// dep[i]表示顶点i的深度,n个顶点,m个询问,rt为树根,fa数组用来预处理顶点i向上跳2^j步之后的顶点
int dep[maxn], n, m, rt, fa[maxn][20];
int v[maxn]={0}; // 是否访问标记

// pre是父顶点,rt是当前顶点
void dfs(int pre, int rt)
{
	dep[rt] = dep[pre]+1; // 当前顶点的深度为父顶点加一
	fa[rt][0] = pre; // 当前顶点向上跳一步为父顶点
	v[rt] = 1; // 访问
	// dp预处理
	for(int i=1;i<=19;i++)
	fa[rt][i] = fa[fa[rt][i-1]][i-1];
	// 继续dfs
	for(int i=0;i<son[rt].size();i++)
	if(v[son[rt][i]]==0)
	dfs(rt, son[rt][i]);
}

// 求解LCA(a, b)
int lca(int a, int b)
{
	if(dep[a] < dep[b])
		swap(a, b);
	for(int i=19;i>=0;i--)
	{
		if(dep[a]-dep[b] >= (1<<i))
		{
			a = fa[a][i];
		}
	}
	if(a==b)return a;
	for(int i=19;i>=0;i--)
	{
		if(fa[a][i] != fa[b][i])
		{
			a = fa[a][i];
			b = fa[b][i];
		}
	}
	return fa[a][0];
}

int main()
{
	scanf("%d%d%d", &n, &m, &rt);
	for(int i=1;i<n;i++)
	{
		int a, b;
		scanf("%d%d", &a, &b);
		son[a].push_back(b);
		son[b].push_back(a);
	}
	memset(fa, 0, sizeof(fa));
	memset(dep, inf, sizeof(dep));
	v[0]=1;
	dep[0] = 0;
	dfs(0, rt);
	for(int i=1;i<=m;i++)
	{
		int a, b;
		scanf("%d%d", &a, &b);
		printf("%d\n", lca(a, b));
	}
	return 0;
}
\end{lstlisting}

\subsubsection{RMQ}
\begin{lstlisting}
#include<bits/stdc++.h>
using namespace std;
typedef long long ll;
#define inf 0x3f3f3f3f
typedef pair<int, int> P;
const int maxn = 5e5+5;
const ll mod = 1e9+7;

vector<int> g[maxn]; // 存图
// dep记录DFS序中每一个顶点的深度, vis记录DFS序, id记录顶点i第一次在DFS序中的位置, st表
int dep[maxn<<1]={0}, vis[maxn<<1]={0}, id[maxn]={0}, st[maxn<<1][25];
// dfs序计数用,看代码能理解
int dfs_c=1;

// 父顶点为pre, 当前顶点为now, 当前深度为d
void dfs(int pre, int now, int d)
{
	id[now] =  dfs_c; // now顶点在DFS序中第一次出现的位置是dfs_c
	dep[dfs_c] = d; // 记录now的深度
	vis[dfs_c++] = now; // DFS序中第dfs_c个顶点是now,同时将dfs_c加一
	for(int i=0;i<g[now].size();i++)
	{
		if(g[now][i]!=pre)
		{
			dfs(now, g[now][i], d+1);
			vis[dfs_c] = now;
			dep[dfs_c++] = d;
		}
	}
}

// 预处理st表
void getSt(int n)
{
	for(int i=1;i<=n;i++)
	st[i][0] = i;
	for(int j=1; (1<<j)<=n; j++)
	{
		for(int i=1;i+(1<<j)<=n; i++)
		{
			int a = st[i][j-1], b = st[i+(1<<(j-1))][j-1];
			if(dep[a] < dep[b])
				st[i][j] = a;
			else st[i][j] = b;
		}
	}
}

// 查询DFS序中区间[l, r]深度最小的顶点在DFS序中的位置
int query(int l, int r)
{
	int k = log2(r-l+1);
	int a = st[l][k];
	int b = st[r-(1<<k)+1][k];
	// 返回深度较小的那一个顶点在DFS序中的位置
	if(dep[a]<dep[b])return a;
	else return b;
}

// 求LCA(a, b)
int lca(int a, int b)
{
	int x, y;
	x = id[a], y = id[b];
	if(x>y)return vis[query(y, x)];
	else return vis[query(x, y)];
}

// 检查用的
void check(int n)
{
	for(int i=1;i<=dfs_c;i++)cout<<dep[i]<<" ";cout<<"\n\n";
	for(int i=1;i<=dfs_c;i++)cout<<vis[i]<<" ";cout<<"\n\n";
	for(int i=1;i<=n;i++)cout<<id[i]<<" ";cout<<"\n\n";
}

int main()
{
	int n, m, rt;
	scanf("%d%d%d", &n, &m, &rt);
	for(int i=1;i<n;i++)
	{
		int a, b;
		scanf("%d%d", &a, &b);
		g[a].push_back(b);
		g[b].push_back(a);
	}
	dfs(0, rt, 1);
	getSt(dfs_c);
	//check(n);
	for(int i=1;i<=m;i++)
	{
		int a, b;
		scanf("%d%d", &a, &b);
		printf("%d\n", lca(a, b));
	}
	return 0;
}
\end{lstlisting}


\subsection{强连通分量}
\subsubsection{Tarjan}
\begin{lstlisting}
vector<int> g[maxn];
int low[maxn], dfn[maxn], sta[maxn], ins[maxn], belong[maxn];
int cnt, ind, tot; //cnt:强连通分量的数量, ind:时间戳, tot:sta的top

void init()
{
	memset(ins, 0, sizeof(ins));
	memset(belong, 0, sizeof(belong));
	memset(dfn, 0, sizeof(dfn));
	cnt = ind = tot = 0;
}

void Tarjan(int u)
{
	low[u] = dfn[u] = ++ind;
	ins[u] = 1;
	sta[++tot] = u;
	for(int i=0;i<g[u].size();i++)
	{
		int v = g[u][i];
		if(!dfn[v])
		{
			Tarjan(v);
			low[u] = min(low[u], low[v]);
		}
		else if(ins[v])
		low[u] = min(low[u], dfn[v]);
	}
	int p;
	if(low[u] == dfn[u])
	{
		++cnt;
		do
		{
			p = sta[tot--];
			belong[p] = cnt;
			ins[p] = 0;
		}while(p != u);
	}
}
\end{lstlisting}

\subsection{割点}
\subsubsection{Tarjan}
\begin{lstlisting}
vector<int> g[maxn];
// iscut[i]: 若顶点i是割点,则为1,反之为0
int low[maxn], dfn[maxn], iscut[maxn];
int ind;

void init()
{
	memset(dfn, 0, sizeof(dfn));
	memset(iscut, 0, sizeof(iscut));
	ind = 0;
}

// pa为u的父节点,初始时Tarjan(i, i)
void Tarjan(int u, int pa)
{
	int cnt = 0; //用来记录子树的数量
	low[u] = dfn[u] = ++ind;
	for(int i=0;i<g[u].size();i++)
	{
		int v = g[u][i];
		if(!dfn[v])
		{
			Tarjan(v, u);
			low[u] = min(low[u], low[v]);
			// 若low[v]>=dfn[u],并且u不是根节点,则u是割点
			if(low[v] >= dfn[u] && pa!=u)
			iscut[u] = 1;
			// 若u是根节点,则cnt++
			if(u == pa)
				cnt++;
		}
		else if(v != pa) //若v不等于父节点
		low[u] = min(low[u], dfn[v]);
	}
	if(cnt>=2 && u==pa) //根节点子树数量大于等于2,则为割点
		iscut[u] = 1;
}
\end{lstlisting}

\subsection{桥}
\subsubsection{Tarjan}
\begin{lstlisting}
// 用链式前向星来存储边
struct Edge
{
	// iscut表示是否为桥
	int to, next, iscut;
}e[maxn*maxn*2];

int head[maxn], low[maxn], dfn[maxn];
int ind, tot; // tot是边的数量

void init()
{
	memset(head, -1, sizeof(head));
	memset(dfn, 0, sizeof(dfn));
	ind = tot = 0;
}

void addedge(int u, int v)
{
	e[tot].to = v;
	e[tot].next = head[u];
	e[tot].iscut = 0;
	head[u] = tot++;
}

void Tarjan(int u, int pa)
{
	low[u] = dfn[u] = ++ind;
	for(int i=head[u]; ~i; i = e[i].next)
	{
		int v = e[i].to;
		if(v == pa) continue;
		if(!dfn[v])
		{
			Tarjan(v, u);
			low[u] = min(low[u], low[v]);
			// 是桥
			if(low[v] > dfn[u])
			{
				e[i].iscut = e[i^1].iscut = 1;
			}
		}
		else
		{
			low[u] = min(low[u], dfn[v]);
		}
	}
}
\end{lstlisting}


\subsection{最大流}
\subsubsection{Dinic}
\begin{lstlisting}
#include<bits/stdc++.h>
using namespace std;
typedef long long ll;
typedef pair<int, int> P;
const int maxn = 1e6+5;
const int inf = 0x3f3f3f3f;
const int mod = 1e9+7;

// 用链式前向星来存储图
struct ed
{
	int to, val, ne;
}edge[maxn<<1];
int head[maxn], dep[maxn];
// 顶点数n,边数m,源点s,汇点e,加边时的指针tot
int n, m, s, e, tot;

void init()
{
	tot = -1;
	memset(head, -1, sizeof(head));
}

void addEdge(int u, int v, int val)
{
	edge[++tot].to = v;
	edge[tot].val = val;
	edge[tot].ne = head[u];
	head[u] = tot;
}

// 就是最普通的bfs
int bfs()
{
	memset(dep, -1, sizeof(dep));
	dep[s] = 0;
	queue<int> q;
	q.push(s);
	while(!q.empty())
	{ 
		int u = q.front();
		q.pop();
		for(int i=head[u]; ~i; i=edge[i].ne)
		{
			int v = edge[i].to;
			if(dep[v]==-1 && edge[i].val>0)
			{
				dep[v] = dep[u]+1;
				q.push(v);
			}
		}
	}
	return (dep[e] != -1); //若dep[e]==-1则表示没有可以到达e的增广路了,算法结束。
}

// 当前顶点u,当前流量flow
// 初始时dfs(s, inf)
int dfs(int u, int flow)
{
	if(u == e)return flow;
	for(int i=head[u]; ~i; i=edge[i].ne)
	{
		int v = edge[i].to;
		if(dep[v]==dep[u]+1 && edge[i].val)
		{
			int a = dfs(v, min(flow, edge[i].val));
			if(a>0) //若找到增广路
			{
				edge[i].val -= a;
				edge[i^1].val += a;
				return a;
			}        
		}
	}
	return 0;
}

ll dinic()
{
	ll ans = 0;
	while(bfs())
	{
		int a = dfs(s, (1<<30));
		ans += a;
	}
	return ans;
}

int main()
{
	scanf("%d%d%d%d", &n, &m, &s, &e);
	init();
	for(int i=1;i<=m;i++)
	{
		int u, v, w;
		scanf("%d%d%d", &u, &v, &w);
		addEdge(u, v, w);
		addEdge(v, u, 0); //反边
	}
	printf("%lld\n", dinic());
	return 0;
}
\end{lstlisting}

\subsubsection{Dinic优化}
\begin{lstlisting}
#include<bits/stdc++.h>
using namespace std;
typedef long long ll;
typedef pair<int, int> P;
const int maxn = 1e6+5;
const int inf = 0x3f3f3f3f;
const int mod = 1e9+7;

struct ed
{
	int to, val, ne;
}edge[maxn<<1];
int head[maxn],dep[maxn], cur[maxn];
int n, m, s, e, tot;

void init()
{
	tot = -1;
	memset(head, -1, sizeof(head));
}

void addEdge(int u, int v, int val)
{
	edge[++tot].to = v;
	edge[tot].val = val;
	edge[tot].ne = head[u];
	head[u] = tot;
}

int bfs()
{
	memset(dep, -1, sizeof(dep));
	dep[s] = 0;
	queue<int> q;
	q.push(s);
	while(!q.empty())
	{ 
		int u = q.front();
		q.pop();
		for(int i=head[u]; ~i; i=edge[i].ne)
		{
			int v = edge[i].to;
			if(dep[v]==-1 && edge[i].val>0)
			{
				dep[v] = dep[u]+1;
				q.push(v);
			}
		}
	}
	return (dep[e] != -1);
}

int dfs(int u, int flow)
{
	if(u == e)return flow;
	// rflow用于多路增广,表示流入到顶点u的剩余未流出的流量
	int rflow = flow;
	// 当前弧优化,通过引用,可以改变cur[i]的值,使得下次遍历到顶点u时,会直接从上次增广的边开始遍历
	for(int& i=cur[u]; ~i; i=edge[i].ne)
	{
		int v = edge[i].to;
		if(dep[v]==dep[u]+1 && edge[i].val)
		{
			int a = dfs(v, min(rflow, edge[i].val));
			edge[i].val -= a;
			edge[i^1].val += a;
			rflow -= a; // 剩余流量要减少
			if(rflow<=0)break; // 若没有剩余流量了,就break
		}
	}
	// 若没有一丝流量流出,则表示通过顶点u已经无法增广了,于是炸点,dep可以设置为任何无意义值
	if(rflow == flow)
	dep[u] = -2;
	return flow - rflow; // 返回流出的流量
}

ll dinic()
{
	ll ans = 0;
	while(bfs())
	{
		// 新一轮dfs之前要对cur进行初始化
		for(int i=1;i<=n;i++)cur[i] = head[i];
		int a = dfs(s, (1<<30));
		ans += a;
	}
	return ans;
}

int main()
{
	scanf("%d%d%d%d", &n, &m, &s, &e);
	init();
	for(int i=1;i<=m;i++)
	{
		int u, v, w;
		scanf("%d%d%d", &u, &v, &w);
		addEdge(u, v, w);
		addEdge(v, u, 0);
	}
	printf("%lld\n", dinic());
	return 0;
}
\end{lstlisting}

\section{数据结构}


\section{DP}


\section{字符串}


\section{数学}


\section{STL}


\section{计算几何}


\section{其它}