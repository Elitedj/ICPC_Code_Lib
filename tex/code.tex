\section{图论}

\subsection{最短路径}
\subsubsection{Dijkstra}
\begin{lstlisting}
const int maxn = 1e4;
const int inf = 0x3f3f3f3f;

//d数组用来记录源点s到顶点i的最短距离
//v表示该顶点是否在顶点集S中
//g邻接矩阵存图,g[i][j]表示i到j的边的权值,无边时为inf
//n为顶点数量
int d[maxn], v[maxn];
int g[maxn][maxn];
int n;
void dij(int s)
{
	memset(v, 0, sizeof(v));
	for(int i=1;i<=n;i++)
	d[i] = g[s][i];
	v[s] = 1;
	for(int i=1;i<=n;i++)
	{
		int u = 0;
		for(int j=1;j<=n;j++)
		{
			if(!v[j] && (u==0 || d[j] < d[u]))
				u = j;
		}
		if(u==0)return ;
		v[u] = 1;
		for(int j=1;j<=n;j++)
		{
			d[j] = min(d[j], d[u]+g[u][j]);
		}
	}
}
\end{lstlisting}

\subsubsection{Dijkstra优化}
\begin{lstlisting}
const int maxn = 1e4;
const int inf = 0x3f3f3f3f;
typedef pair<int, int> P; //first表示最短距离,second表示顶点编号
//边:to表示这条边指向的顶点,权值为w
struct Edge
{
	int to, w;
};
//用vector实现邻接表
vector<Edge> g[maxn];
int d[maxn]; //记录源点到顶点i的最短距离
int n;

void dij(int s)
{
	priority_queue<P, vector<P>, greater<P> > q;
	memset(d, inf, sizeof(d));
	d[s] = 0;
	q.push(P(0, s));
	while(!q.empty())
	{
		P p = q.top();
		q.pop();
		int u = p.second;
		if(d[u] < p.first) continue;
		for(int i=0; i<g[u].size(); i++)
		{
			Edge e = g[u][i];
			if(d[e.to] > d[u] + e.w)
			{
				d[e.to] = d[u] + e.w;
				q.push(P(d[e.to], e.to));
			}
		}
	}
}
\end{lstlisting}

\subsubsection{Floyd}
\begin{lstlisting}
int g[maxn][maxn];
int n;
void floyd()
{
	for(int k=1;k<=n;k++)
		for(int i=1;i<=n;i++)
			for(int j=1;j<=n;j++)
				g[i][j] = min(g[i][j], g[i][k] + g[k][j]);
}
\end{lstlisting}

\subsubsection{Bellman-Ford}
\begin{lstlisting}
const int maxn = 1e4;
const int inf = 0x3f3f3f3f; //常用于表示无穷大

//边结构体,记录u->v的边,权值为w
struct Edge
{
	int u, v, w;
	Edge(int uu, int vv, int ww) { u=uu; v=vv; w=ww; }
	Edge(){}
}e[maxn];
int edgecnt; // 边的数量
//加边操作
void addEdge(int u, int v, int w)
{
	e[edgecnt++] = Edge(u, v, w);
}

int n; //顶点总数
int d[maxn]; //记录最短距离的数组

//存在负权回路则返回true,否则返回false
bool bellman_ford(int s)
{
	memset(d, inf, sizeof(d));
	d[s] = 0;
	//进行n-1次松弛操作,第n次检查是否含有负权回路
	for(int i=1;i<=n;i++)
	{
		int flag = 0;
		for(int j=0; j<edgecnt; j++)
		{
			Edge t = e[j];
			int u, v, w;
			u = t.u; v = t.v; w = t.w;
			if(d[v] > d[u] + w)
			{
				d[v] = d[u] + w;
				flag = 1;
			}
		}
		if(!flag) return false;
		if(i==n && flag) return true;
	}
	return false;
}
\end{lstlisting}

\subsubsection{SPFA}
\begin{lstlisting}
const int maxn = 1e4;
const int inf = 0x3f3f3f3f; //常用于表示无穷大

//边结构体,to表示边指向的顶点编号,权值为w
struct Edge
{
	int to, w;
	Edge(int tt, int ww) { to = tt; w = ww; }
	Edge(){}
};
//vector实现的邻接表
vector<Edge> g[maxn];
int n;//顶点数
//d表示最短距离, inq[i]表示结点是否在队列中,为1则在,cnt[i]记录i入队的次数
int d[maxn], inq[maxn], cnt[maxn];
//初始化
void init()
{
	memset(d, inf, sizeof(d));
	memset(inq, 0, sizeof(inq));
	memset(cnt, 0, sizeof(cnt));
}
//返回true表示存在负权回路
bool spfa(int s)
{
	init();
	d[s] = 0;
	inq[s] = 1;
	cnt[s] = 1;
	queue<int> q;
	q.push(s);
	while(!q.empty())
	{
		int u = q.front();
		inq[u] = 0;
		q.pop();
		for(int i=0;i < g[u].size(); i++)
		{
			Edge e = g[u][i];
			if(d[e.to] > d[u] + e.w)
			{
				d[e.to] = d[u] + e.w;
				if(inq[e.to] == 0)
				{
					inq[e.to] = 1;
					q.push(e.to);
					cnt[e.to]++;
					if(cnt[e.to] > n) return true;
				}
			}
		}
	}
	return true;
}
\end{lstlisting}

\subsection{次短路}
\begin{lstlisting}
#include <bits/stdc++.h>
#define INF 1e16+100
#define ms(x,y) memset(x,y,sizeof(x))
using namespace std;

typedef long long ll;
typedef pair<ll,ll> P;

const double pi = acos(-1.0);
const int mod = 1e9 + 7;
const int maxn = 1e5 + 5;

struct Edge{
	ll to,cost;
};

ll n,m;
vector<Edge> a[maxn];
ll dist[maxn],dist2[maxn];

void addedge(ll u,ll v,ll w)
{
	a[u].push_back(Edge{v,w});
	a[v].push_back(Edge{u,w});
}

void solve()
{
	priority_queue<P, vector<P>, greater<P> >que;
	//ms(dist,INF);
	//ms(dist2,INF);
	fill(dist,dist+n,INF);
	fill(dist2,dist2+n,INF);
	dist[0]=0;
	que.push(P(0,0));
	while(que.size())
	{
		P u=que.top();que.pop();
		int v=u.second;
		ll d=u.first;
		if(dist2[v]<d) continue;	//不是次短距离则抛弃
		for(int i=0;i<a[v].size();i++)
		{
			Edge e=a[v][i];
			ll d2=d+e.cost;
			if(dist[e.to]>d2)	//更新最短
			{
				swap(dist[e.to],d2);
				que.push(P(dist[e.to],e.to));
			}
			if(dist2[e.to]>d2&&dist[e.to]<d2)	//更新次短
			{
				dist2[e.to]=d2;
				que.push(P(dist2[e.to],e.to));
			}
		}
	}
	printf("%lld\n",dist2[n-1]);
}

int main()
{
	//freopen("in.txt","r",stdin);
	//freopen("out.txt","w",stdout);
	int t;
	scanf("%d",&t);
	while(t--)
	{
		scanf("%lld%lld",&n,&m);
		for(int i=0;i<n;i++) a[i].clear();
		for(int i=0;i<m;i++)
		{
			ll p,q,w;
			scanf("%lld%lld%lld",&p,&q,&w);
			addedge(p-1,q-1,w);
		}
		solve();
	}
	return 0;
}
\end{lstlisting}

\subsection{LCA}
\subsubsection{倍增}
\begin{lstlisting}
#include<bits/stdc++.h>
using namespace std;
typedef long long ll;
#define inf 0x3f3f3f3f
typedef pair<int, int> P;
const int maxn = 5e5+5;
const ll mod = 1e9+7;

vector<int> son[maxn]; // 存储儿子顶点
// dep[i]表示顶点i的深度,n个顶点,m个询问,rt为树根,fa数组用来预处理顶点i向上跳2^j步之后的顶点
int dep[maxn], n, m, rt, fa[maxn][20];
int v[maxn]={0}; // 是否访问标记

// pre是父顶点,rt是当前顶点
void dfs(int pre, int rt)
{
	dep[rt] = dep[pre]+1; // 当前顶点的深度为父顶点加一
	fa[rt][0] = pre; // 当前顶点向上跳一步为父顶点
	v[rt] = 1; // 访问
	// dp预处理
	for(int i=1;i<=19;i++)
	fa[rt][i] = fa[fa[rt][i-1]][i-1];
	// 继续dfs
	for(int i=0;i<son[rt].size();i++)
	if(v[son[rt][i]]==0)
	dfs(rt, son[rt][i]);
}

// 求解LCA(a, b)
int lca(int a, int b)
{
	if(dep[a] < dep[b])
		swap(a, b);
	for(int i=19;i>=0;i--)
	{
		if(dep[a]-dep[b] >= (1<<i))
		{
			a = fa[a][i];
		}
	}
	if(a==b)return a;
	for(int i=19;i>=0;i--)
	{
		if(fa[a][i] != fa[b][i])
		{
			a = fa[a][i];
			b = fa[b][i];
		}
	}
	return fa[a][0];
}

int main()
{
	scanf("%d%d%d", &n, &m, &rt);
	for(int i=1;i<n;i++)
	{
		int a, b;
		scanf("%d%d", &a, &b);
		son[a].push_back(b);
		son[b].push_back(a);
	}
	memset(fa, 0, sizeof(fa));
	memset(dep, inf, sizeof(dep));
	v[0]=1;
	dep[0] = 0;
	dfs(0, rt);
	for(int i=1;i<=m;i++)
	{
		int a, b;
		scanf("%d%d", &a, &b);
		printf("%d\n", lca(a, b));
	}
	return 0;
}
\end{lstlisting}

\subsubsection{RMQ}
\begin{lstlisting}
#include<bits/stdc++.h>
using namespace std;
typedef long long ll;
#define inf 0x3f3f3f3f
typedef pair<int, int> P;
const int maxn = 5e5+5;
const ll mod = 1e9+7;

vector<int> g[maxn]; // 存图
// dep记录DFS序中每一个顶点的深度, vis记录DFS序, id记录顶点i第一次在DFS序中的位置, st表
int dep[maxn<<1]={0}, vis[maxn<<1]={0}, id[maxn]={0}, st[maxn<<1][25];
// dfs序计数用,看代码能理解
int dfs_c=1;

// 父顶点为pre, 当前顶点为now, 当前深度为d
void dfs(int pre, int now, int d)
{
	id[now] =  dfs_c; // now顶点在DFS序中第一次出现的位置是dfs_c
	dep[dfs_c] = d; // 记录now的深度
	vis[dfs_c++] = now; // DFS序中第dfs_c个顶点是now,同时将dfs_c加一
	for(int i=0;i<g[now].size();i++)
	{
		if(g[now][i]!=pre)
		{
			dfs(now, g[now][i], d+1);
			vis[dfs_c] = now;
			dep[dfs_c++] = d;
		}
	}
}

// 预处理st表
void getSt(int n)
{
	for(int i=1;i<=n;i++)
	st[i][0] = i;
	for(int j=1; (1<<j)<=n; j++)
	{
		for(int i=1;i+(1<<j)<=n; i++)
		{
			int a = st[i][j-1], b = st[i+(1<<(j-1))][j-1];
			if(dep[a] < dep[b])
				st[i][j] = a;
			else st[i][j] = b;
		}
	}
}

// 查询DFS序中区间[l, r]深度最小的顶点在DFS序中的位置
int query(int l, int r)
{
	int k = log2(r-l+1);
	int a = st[l][k];
	int b = st[r-(1<<k)+1][k];
	// 返回深度较小的那一个顶点在DFS序中的位置
	if(dep[a]<dep[b])return a;
	else return b;
}

// 求LCA(a, b)
int lca(int a, int b)
{
	int x, y;
	x = id[a], y = id[b];
	if(x>y)return vis[query(y, x)];
	else return vis[query(x, y)];
}

// 检查用的
void check(int n)
{
	for(int i=1;i<=dfs_c;i++)cout<<dep[i]<<" ";cout<<"\n\n";
	for(int i=1;i<=dfs_c;i++)cout<<vis[i]<<" ";cout<<"\n\n";
	for(int i=1;i<=n;i++)cout<<id[i]<<" ";cout<<"\n\n";
}

int main()
{
	int n, m, rt;
	scanf("%d%d%d", &n, &m, &rt);
	for(int i=1;i<n;i++)
	{
		int a, b;
		scanf("%d%d", &a, &b);
		g[a].push_back(b);
		g[b].push_back(a);
	}
	dfs(0, rt, 1);
	getSt(dfs_c);
	//check(n);
	for(int i=1;i<=m;i++)
	{
		int a, b;
		scanf("%d%d", &a, &b);
		printf("%d\n", lca(a, b));
	}
	return 0;
}
\end{lstlisting}


\subsection{强连通分量}
\begin{lstlisting}
vector<int> g[maxn];
int low[maxn], dfn[maxn], sta[maxn], ins[maxn], belong[maxn];
int cnt, ind, tot; //cnt:强连通分量的数量, ind:时间戳, tot:sta的top

void init()
{
	memset(ins, 0, sizeof(ins));
	memset(belong, 0, sizeof(belong));
	memset(dfn, 0, sizeof(dfn));
	cnt = ind = tot = 0;
}

void Tarjan(int u)
{
	low[u] = dfn[u] = ++ind;
	ins[u] = 1;
	sta[++tot] = u;
	for(int i=0;i<g[u].size();i++)
	{
		int v = g[u][i];
		if(!dfn[v])
		{
			Tarjan(v);
			low[u] = min(low[u], low[v]);
		}
		else if(ins[v])
		low[u] = min(low[u], dfn[v]);
	}
	int p;
	if(low[u] == dfn[u])
	{
		++cnt;
		do
		{
			p = sta[tot--];
			belong[p] = cnt;
			ins[p] = 0;
		}while(p != u);
	}
}
\end{lstlisting}

\subsection{割点}
\begin{lstlisting}
vector<int> g[maxn];
// iscut[i]: 若顶点i是割点,则为1,反之为0
int low[maxn], dfn[maxn], iscut[maxn];
int ind;

void init()
{
	memset(dfn, 0, sizeof(dfn));
	memset(iscut, 0, sizeof(iscut));
	ind = 0;
}

// pa为u的父节点,初始时Tarjan(i, i)
void Tarjan(int u, int pa)
{
	int cnt = 0; //用来记录子树的数量
	low[u] = dfn[u] = ++ind;
	for(int i=0;i<g[u].size();i++)
	{
		int v = g[u][i];
		if(!dfn[v])
		{
			Tarjan(v, u);
			low[u] = min(low[u], low[v]);
			// 若low[v]>=dfn[u],并且u不是根节点,则u是割点
			if(low[v] >= dfn[u] && pa!=u)
			iscut[u] = 1;
			// 若u是根节点,则cnt++
			if(u == pa)
				cnt++;
		}
		else if(v != pa) //若v不等于父节点
		low[u] = min(low[u], dfn[v]);
	}
	if(cnt>=2 && u==pa) //根节点子树数量大于等于2,则为割点
		iscut[u] = 1;
}
\end{lstlisting}

\subsection{桥}
\begin{lstlisting}
// 用链式前向星来存储边
struct Edge
{
	// iscut表示是否为桥
	int to, next, iscut;
}e[maxn*maxn*2];

int head[maxn], low[maxn], dfn[maxn];
int ind, tot; // tot是边的数量

void init()
{
	memset(head, -1, sizeof(head));
	memset(dfn, 0, sizeof(dfn));
	ind = tot = 0;
}

void addedge(int u, int v)
{
	e[tot].to = v;
	e[tot].next = head[u];
	e[tot].iscut = 0;
	head[u] = tot++;
}

void Tarjan(int u, int pa)
{
	low[u] = dfn[u] = ++ind;
	for(int i=head[u]; ~i; i = e[i].next)
	{
		int v = e[i].to;
		if(v == pa) continue;
		if(!dfn[v])
		{
			Tarjan(v, u);
			low[u] = min(low[u], low[v]);
			// 是桥
			if(low[v] > dfn[u])
			{
				e[i].iscut = e[i^1].iscut = 1;
			}
		}
		else
		{
			low[u] = min(low[u], dfn[v]);
		}
	}
}
\end{lstlisting}


\subsection{最大流}
\subsubsection{Dinic}
\begin{lstlisting}
#include<bits/stdc++.h>
using namespace std;
typedef long long ll;
typedef pair<int, int> P;
const int maxn = 1e6+5;
const int inf = 0x3f3f3f3f;
const int mod = 1e9+7;

// 用链式前向星来存储图
struct ed
{
	int to, val, ne;
}edge[maxn<<1];
int head[maxn], dep[maxn];
// 顶点数n,边数m,源点s,汇点e,加边时的指针tot
int n, m, s, e, tot;

void init()
{
	tot = -1;
	memset(head, -1, sizeof(head));
}

void addEdge(int u, int v, int val)
{
	edge[++tot].to = v;
	edge[tot].val = val;
	edge[tot].ne = head[u];
	head[u] = tot;
}

// 就是最普通的bfs
int bfs()
{
	memset(dep, -1, sizeof(dep));
	dep[s] = 0;
	queue<int> q;
	q.push(s);
	while(!q.empty())
	{ 
		int u = q.front();
		q.pop();
		for(int i=head[u]; ~i; i=edge[i].ne)
		{
			int v = edge[i].to;
			if(dep[v]==-1 && edge[i].val>0)
			{
				dep[v] = dep[u]+1;
				q.push(v);
			}
		}
	}
	return (dep[e] != -1); //若dep[e]==-1则表示没有可以到达e的增广路了,算法结束。
}

// 当前顶点u,当前流量flow
// 初始时dfs(s, inf)
int dfs(int u, int flow)
{
	if(u == e)return flow;
	for(int i=head[u]; ~i; i=edge[i].ne)
	{
		int v = edge[i].to;
		if(dep[v]==dep[u]+1 && edge[i].val)
		{
			int a = dfs(v, min(flow, edge[i].val));
			if(a>0) //若找到增广路
			{
				edge[i].val -= a;
				edge[i^1].val += a;
				return a;
			}        
		}
	}
	return 0;
}

ll dinic()
{
	ll ans = 0;
	while(bfs())
	{
		int a = dfs(s, (1<<30));
		ans += a;
	}
	return ans;
}

int main()
{
	scanf("%d%d%d%d", &n, &m, &s, &e);
	init();
	for(int i=1;i<=m;i++)
	{
		int u, v, w;
		scanf("%d%d%d", &u, &v, &w);
		addEdge(u, v, w);
		addEdge(v, u, 0); //反边
	}
	printf("%lld\n", dinic());
	return 0;
}
\end{lstlisting}

\subsubsection{Dinic优化}
\begin{lstlisting}
#include<bits/stdc++.h>
using namespace std;
typedef long long ll;
typedef pair<int, int> P;
const int maxn = 1e6+5;
const int inf = 0x3f3f3f3f;
const int mod = 1e9+7;

struct ed
{
	int to, val, ne;
}edge[maxn<<1];
int head[maxn],dep[maxn], cur[maxn];
int n, m, s, e, tot;

void init()
{
	tot = -1;
	memset(head, -1, sizeof(head));
}

void addEdge(int u, int v, int val)
{
	edge[++tot].to = v;
	edge[tot].val = val;
	edge[tot].ne = head[u];
	head[u] = tot;
}

int bfs()
{
	memset(dep, -1, sizeof(dep));
	dep[s] = 0;
	queue<int> q;
	q.push(s);
	while(!q.empty())
	{ 
		int u = q.front();
		q.pop();
		for(int i=head[u]; ~i; i=edge[i].ne)
		{
			int v = edge[i].to;
			if(dep[v]==-1 && edge[i].val>0)
			{
				dep[v] = dep[u]+1;
				q.push(v);
			}
		}
	}
	return (dep[e] != -1);
}

int dfs(int u, int flow)
{
	if(u == e)return flow;
	// rflow用于多路增广,表示流入到顶点u的剩余未流出的流量
	int rflow = flow;
	// 当前弧优化,通过引用,可以改变cur[i]的值,使得下次遍历到顶点u时,会直接从上次增广的边开始遍历
	for(int& i=cur[u]; ~i; i=edge[i].ne)
	{
		int v = edge[i].to;
		if(dep[v]==dep[u]+1 && edge[i].val)
		{
			int a = dfs(v, min(rflow, edge[i].val));
			edge[i].val -= a;
			edge[i^1].val += a;
			rflow -= a; // 剩余流量要减少
			if(rflow<=0)break; // 若没有剩余流量了,就break
		}
	}
	// 若没有一丝流量流出,则表示通过顶点u已经无法增广了,于是炸点,dep可以设置为任何无意义值
	if(rflow == flow)
	dep[u] = -2;
	return flow - rflow; // 返回流出的流量
}

ll dinic()
{
	ll ans = 0;
	while(bfs())
	{
		// 新一轮dfs之前要对cur进行初始化
		for(int i=1;i<=n;i++)cur[i] = head[i];
		int a = dfs(s, (1<<30));
		ans += a;
	}
	return ans;
}

int main()
{
	scanf("%d%d%d%d", &n, &m, &s, &e);
	init();
	for(int i=1;i<=m;i++)
	{
		int u, v, w;
		scanf("%d%d%d", &u, &v, &w);
		addEdge(u, v, w);
		addEdge(v, u, 0);
	}
	printf("%lld\n", dinic());
	return 0;
}
\end{lstlisting}


\subsection{二分图匹配}
\subsubsection{匈牙利算法}
\begin{lstlisting}
#include<bits/stdc++.h>
using namespace std;
typedef long long ll;
typedef pair<int, int> P;
const int maxn = 1e3+5;
const int inf = 0x3f3f3f3f;
const int mod = 1e9+7;


int mp[maxn][maxn];
int use[maxn], link[maxn];
int n, m, e, ans;

int found(int u)
{
	for(int i=1;i<=m;i++)
	{
		if(!use[i] && mp[u][i])
		{
			use[i] = 1;
			if(!link[i] || found(link[i]))
			{
				link[i] = u;
				return 1;
			}
		}
	}
	return 0;
}

int main()
{
	scanf("%d%d%d", &n, &m, &e);
	for(int i=1;i<=e;i++)
	{
		int u, v;
		scanf("%d%d", &u, &v);
		if(u<=n && v<=m)
		{
			mp[u][v] = 1;
		}
	}
	for(int i=1;i<=n;i++)
	{
		memset(use, 0, sizeof(use));
		if(found(i))ans++;
	}
	printf("%d\n", ans);
	return 0;
}
\end{lstlisting}


\subsection{最小生成树}
\subsubsection{Prim}
\begin{lstlisting}
const int MAX=10000007;
int dis[5002],map[5002][5002],mark[5002];
int prim(int n)
{
	for(int i=1;i<=n;i++)  //初始化每个点到生成树中点的距离
	{
		dis[i]=map[1][i];
		mark[i]=0;
	}
	dis[1]=0;
	mark[1]=1; //1这个点加入生成树中。
	int sum=0;
	for(int i=1;i<n;i++) //枚举n-1条边
	{
		int sta=-1,Min=MAX;
		for(int j=1;j<=n;j++)  //找不在生成树中的点中距离生成树中的点长度最小的
		{
			if(!mark[j]&&dis[j]<Min)
			{
				Min=dis[j];
				sta=j;
			}
		}
		if(sta==-1) return -1; //没找到可以可以联通的路
		mark[sta]=1;   //新找到的点加入生成树
		sum+=Min;
		for(int j=1;j<=n;j++)  //更新树外的点到树中的点的距离
		{
			if(!mark[j]&&dis[j]>map[sta][j])
			dis[j]=map[sta][j];
		}
	}
	return sum;
}

int main()
{
	int n,m;
	cin>>n>>m;
	for(int i=1;i<=n;i++)
	{
		for(int j=1;j<=n;j++)
		{
			map[i][j]=MAX;
		}
	}
	for(int i=1;i<=m;i++)
	{
		int a,b,c;
		cin>>a>>b>>c;
		if(c<map[a][b])
		{
			map[a][b]=c;
			map[b][a]=c;
		}
	}
	int ans = prim(n);
	if(ans==-1)
		cout<<"orz"<<endl;
	else
		cout<<ans<<endl;
	return 0;
}
\end{lstlisting}


\subsubsection{kruskal}
\begin{lstlisting}
#include<bits/stdc++.h>
using namespace std;
#define N 5005
int father[N];

int find(int x)
{
	int k = x;
	while(father[k]!=k)
	{
		k = father[k];
	}
	while(father[x]!=x)
	{
		int temp = x;
		x = father[x];
		father[temp] = k;
	}
	return k;
}

void join(int a, int b)
{
	int f1, f2;
	f1 = find(a);
	f2 = find(b);
	father[f1] = f2;
}

struct edge
{
	int node1, node2;
	int cost;
};

vector<edge> edges;

bool cmp(edge a, edge b)
{
	return a.cost > b.cost;
}

int kruskal(int n)
{
	sort(edges.begin(), edges.end(), cmp);
	for(int i=1;i<=n;i++)
		father[i] = i;
	int sum=0;
	while(n!=1 && !edges.empty())
	{
		edge temp = edges[edges.size()-1];
		edges.pop_back();
		if(find(temp.node1)!=find(temp.node2))
		{
			sum += temp.cost;
			n--;
			join(temp.node1, temp.node2);
		}
	}
	if(n!=1 && edges.empty())
		sum = -1;
	return sum;
}

int main()
{
	int n,m;
	int result;
	cin>>n>>m;
	for(int i=1;i<=m;i++)
	{
		int a,b,c;
		cin>>a>>b>>c;
		edge t;
		t.node1=a;t.node2=b;t.cost=c;
		edges.push_back(t);
	}
	result = kruskal(n);
	if(result == -1)
		cout<<"orz"<<endl;
	else
		cout<<result<<endl;
	return 0;
}
\end{lstlisting}

\subsection{次小生成树-POJ1679}
\begin{lstlisting}
#include<bits/stdc++.h>
using namespace std;
typedef long long ll;
#define inf 0x3f3f3f3f
typedef pair<int, int> P;
const int maxn = 110;
const ll mod = 1e9+7;

int n, m;
int g[maxn][maxn];
int d[maxn], v[maxn], maxd[maxn][maxn], pre[maxn], mst[maxn][maxn];
int ans = 0;

void prim()
{
	for(int i=1;i<=n;i++)
	{
		v[i] = 0; d[i] = inf; pre[i] = 1;
	}
	memset(maxd, 0, sizeof(maxd));
	memset(mst, 0, sizeof(mst));
	ans = 0;
	priority_queue<P, vector<P>, greater<P> > q;
	d[1] = 0; q.push(P(0, 1));
	while(!q.empty())
	{
		P p = q.top(); q.pop();
		int u = p.second;
		if(v[u]) continue;
		v[u] = 1; ans += d[u];
		mst[pre[u]][u] = mst[u][pre[u]] = 1;
		for(int i=1; i<=n;i++)
		{
			if(v[i] && g[u][i] < inf)
				maxd[u][i] = maxd[i][u] = max(maxd[pre[u]][u], d[u]);
			if(d[i] > g[u][i])
			{
				d[i] = g[u][i];
				pre[i] = u;
				q.push(P(d[i], i));
			}
		}
	}
}

int main()
{
	int t;
	cin>>t;
	while(t--)
	{
		memset(g, inf, sizeof(g));
		cin>>n>>m;
		while(m--)
		{
			int a, b, c;
			cin>>a>>b>>c;
			g[a][b] = g[b][a] = c;
		}
		prim();
		int flag = 0;
		for(int i=1;i<=n&&!flag;i++)
		{
			for(int j=1;j<=n;j++)
			{
				if(mst[i][j] || g[i][j]==inf)continue;
				if(g[i][j] == maxd[i][j])
				{
					flag = 1;
					break;
				}
			}
		}
		if(flag) cout<<"Not Unique!"<<endl;
		else cout<<ans<<endl;
	}
	return 0;
}
\end{lstlisting}

\subsection{拓扑排序}
\begin{lstlisting}
#include<bits/stdc++.h>
using namespace std;
const int maxn=30;

int head[maxn],ip,indegree[maxn];
int n,m,seq[maxn];

struct note
{
	int v,next;
}edge[maxn*maxn];

void init()
{
	memset(head,-1,sizeof(head));
	ip=0;
}

void addedge(int u,int v)
{
	edge[ip].v=v,edge[ip].next=head[u],head[u]=ip++;
}

int topo()
{
	queue<int>q;
	int indeg[maxn];
	for(int i=0; i<n; i++)
	{
		indeg[i]=indegree[i];
		if(indeg[i]==0)
		q.push(i);
	}
	int k=0;
	bool res=false;
	while(!q.empty())
	{
		if(q.size()!=1)res=true;
		int u=q.front();
		q.pop();
		seq[k++]=u;
		for(int i=head[u]; i!=-1; i=edge[i].next)
		{
			int v=edge[i].v;
			indeg[v]--;
			if(indeg[v]==0)
				q.push(v);
		}
	}
	if(k<n)return -1;// no
	if(res)return 0;// more
	return 1; // only
}
\end{lstlisting}

\subsection{Floyd找最小环}
\begin{lstlisting}
const int INF = 0x3f3f3f3f;
const int MAXN = 110;

int n, m;               //  n:节点个数, m:边的个数
int g[MAXN][MAXN];      //  无向图
int dist[MAXN][MAXN];   //  最短路径
int r[MAXN][MAXN];      //  r[i][j]: i到j的最短路径的第一步
int out[MAXN], ct;      //  记录最小环

int solve(int i, int j, int k)
{   //  记录最小环
	ct = 0;
	while (j != i)
	{
		out[ct++] = j;
		j = r[i][j];
	}
	out[ct++] = i;
	out[ct++] = k;
	return 0;
}

int main()
{
	while (scanf("%d%d", &n, &m) != EOF)
	{
		int i, j, k;
		for (i = 0; i < n; i++)
		{
			for (j = 0; j < n; j++)
			{
				g[i][j] = INF;
				r[i][j] = i;
			}
		}
		for (i = 0; i < m; i++)
		{
			int x, y, l;
			scanf("%d%d%d", &x, &y, &l);
			--x;
			--y;
			if (l < g[x][y])
			{
				g[x][y] = g[y][x] = l;
			}
		}
		memmove(dist, g, sizeof(dist));
		int Min = INF;              //  最小环
		for (k = 0; k < n; k++)
		{                           //  Floyd
			for (i = 0; i < k; i++) //  一个环中的最大结点为k(编号最大)
			{
				if (g[k][i] < INF)
				{
					for (j = i + 1; j < k; j++)
					{
						if (dist[i][j] < INF && g[k][j] < INF && Min > dist[i][j] + g[k][i] + g[k][j])
						{
							Min = dist[i][j] + g[k][i] + g[k][j];
							solve(i, j, k);     //  记录最小环
						}
					}
				}
			}
			for (i = 0; i < n; i++)
			{
				if (dist[i][k] < INF)
				{
					for (j = 0; j < n; j++)
					{
						if (dist[k][j] < INF && dist[i][j] > dist[i][k]+dist[k][j])
						{
							dist[i][j] = dist[i][k] + dist[k][j];
							r[i][j] = r[k][j];
						}
					}
				}
			}
		}
		if (Min < INF)
		{
			for (ct--; ct >= 0; ct--)
			{
				printf("%d", out[ct] + 1);
				if (ct)
				{
					printf(" ");
				}
			}
		}
		else
		{
			printf("No solution.");
		}
		printf("\n");
	}
	return 0;
}
\end{lstlisting}



\section{数据结构}
\subsection{并查集}
\begin{lstlisting}
int pre[maxn];

int Find(int x)
{
	int p,tmp;
	p=x;
	while(x!=pre[x])
		x=pre[x];
	while(p!=x)
	{
		tmp=pre[x];
		pre[x]=x;
		p=tmp;
	}
	return x;
}

void join(int x,int y)
{
	int fx=Find(x);
	int fy=Find(y);
	if(fx!=fy)
		pre[fx]=fy;
}
\end{lstlisting}

\subsection{ST表}
\begin{lstlisting}
int st[maxn][20];
void st_init()
{
	for(int i=1;i<=n;i++) st[i][0]=a[i]; // 长度为1的区间最小值党委就为自身
	// 预处理从i开始,长度为2^j的区间
	for(int j=1;(1<<j)<=n;j++)
		for(int i=1;i+(1<<j)-1<=n;i++)
			st[i][j]=max(st[i][j-1],st[i+(1<<(j-1))][j-1]);
}

int query(int l,int r)
{
	int k=log2(r-l+1);
	return min(st[l][k],st[r-(1<<k)+1][k]);
}
\end{lstlisting}


\subsection{树状数组}
\begin{lstlisting}
/*
*  INIT: ar[]置为0;
*  CALL: add(i, v): 将i点的值加v; sum(i): 求[1, i]的和;
*/
#define typev int   //  type of res
const int N = 1010;
typev ar[N];        //  index: 1 ~ N
int lowb(int t)
{
	return t & (-t);
}

void add(int i, typev v)
{
	for (; i < N; ar[i] += v, i += lowb(i));
	return ;
}

typev sum(int i)
{
	typev s = 0;
	for (; i > 0; s += ar[i], i -= lowb(i));
	return s;
}
\end{lstlisting}

\subsection{线段树}
\subsubsection{单点修改+区间求和 HDU1166}
\begin{lstlisting}
#include<bits/stdc++.h>
using namespace std;
typedef long long ll;

int num[50005];
ll tree[4 * 50000 + 5];

void build(int p, int l, int r)
{
	if(l == r)
	{
		tree[p] = num[l];
		return ;
	}
	else
	{
		int mid = (r+l) >> 1;
		build(p<<1, l, mid);
		build(p<<1|1, mid+1, r);
		tree[p] = tree[p<<1] + tree[p<<1|1];
	}
}

void add(int p, int l, int r, int ind, int v)
{
	if(l == r)
	{
		tree[p] += v;
		return ;
	}
	else
	{
		int mid = (r+l) >> 1;
		if(ind <= mid) add(p<<1, l, mid, ind, v);
		else add(p<<1|1, mid+1, r, ind, v);
		tree[p] = tree[p<<1] + tree[p<<1|1];
	}
}

ll query(int p, int l, int r, int x, int y)
{
	if(x <= l && r <= y)
	{
		return tree[p];
	}
	else
	{
		int mid = (l+r) >> 1;
		ll ans = 0;
		if(x <= mid)
			ans += query(p<<1, l, mid, x, y);
		if(mid < y)
			ans += query(p<<1|1, mid+1, r, x, y);
		return ans;
	}
}

int main()
{
	int t;
	scanf("%d", &t);
	for(int i=1;i<=t;i++)
	{
		int n;
		scanf("%d", &n);
		for(int j=1;j<=n;j++)
			scanf("%d", &num[j]);
		build(1, 1, n);
		string s;
		printf("Case %d:\n", i);
		while(cin>>s && s[0] != 'E')
		{
			if(s[0] == 'Q')
			{
				int x, y;
				scanf("%d%d", &x, &y);
				printf("%lld\n", query(1, 1, n, x, y));
			}
			else if(s[0] == 'A')
			{
				int x, y;
				scanf("%d%d", &x, &y);
				add(1, 1, n, x, y);
			}
			else
			{
				int x, y;
				scanf("%d%d", &x, &y);
				add(1, 1, n, x, -y);
			}
		}
	}
	return 0;
}
\end{lstlisting}


\subsubsection{区间修改+区间求和 POJ3465}
\begin{lstlisting}
#include<iostream>
using namespace std;
typedef long long ll;
ll num[100005];
ll tree[4 * 100000 + 5];
ll lazy[4 * 100000 + 5]={0};
int n,m;
void build(int p, int l, int r)
{
	if(l == r)
	tree[p] = num[l];
	else
	{
		int mid = (l+r) >> 1;
		build(p<<1, l, mid);
		build(p<<1|1, mid+1, r);
		tree[p]  = tree[p<<1] + tree[p<<1|1];
	}
}

void pushdown(int p, int l, int r)
{
	if(lazy[p])
	{
		lazy[p<<1] += lazy[p];
		lazy[p<<1|1] += lazy[p];
		tree[p<<1] += lazy[p] * (((l+r)>>1) - l + 1);
		tree[p<<1|1] += lazy[p] * (r - ((l+r)>>1));
		lazy[p] = 0;
	}
}

void add(int p, int l, int r, int x, int y, ll v)
{
	if(x <= l && r <= y)
	{
		lazy[p] += v;
		tree[p] += v * (r-l+1);
		return ;
	}
	else
	{
		int mid = (l+r) >> 1;
		pushdown(p, l, r);
		if(x <= mid)
			add(p<<1, l, mid, x, y, v);
		if(y > mid)
			add(p<<1|1, mid+1, r, x, y, v);
		tree[p] = tree[p<<1] + tree[p<<1|1];
	}
}

ll query(int p, int l, int r, int x, int y)
{
	if(x <= l && r <= y)
	{
		return tree[p];
	}
	else
	{
		int mid = (l+r) >> 1;
		ll ans = 0;
		pushdown(p, l, r);
		if(x <= mid)
			ans += query(p<<1, l, mid, x, y);
		if(y > mid)
			ans += query(p<<1|1, mid+1, r, x, y);
		return ans;
	}
}

int main()
{
	cin>>n>>m;
	for(int i=1;i<=n;i++)
	cin>>num[i];
	build(1, 1, n);
	for(int i=1;i<=m;i++)
	{
		char a;
		cin>>a;
		if(a == 'Q')
		{
			int x, y;
			cin>>x>>y;
			cout<<query(1, 1, n, x, y)<<endl;
		}
		else
		{
			int x, y;
			ll c;
			cin>>x>>y>>c;
			add(1, 1, n, x, y, c);
		}
	}
	return 0;
}
\end{lstlisting}

\subsubsection{单点修改+区间最值 HDU1754}
\begin{lstlisting}
#include<bits/stdc++.h>
using namespace std;
typedef long long ll;
int n,m;

int num[200005];
int tree[200000 * 4 + 5];

void build(int p, int l, int r)
{
	if(l == r)
	{
		tree[p] = num[l];
		return ;
	}
	else
	{
		int mid = (l+r) >> 1;
		build(p<<1, l, mid);
		build(p<<1|1, mid+1, r);
		tree[p] = max(tree[p<<1], tree[p<<1|1]);
		return ;
	}
}

void update(int p, int l, int r, int ind, int v)
{
	if(l == r)
	{
		tree[p] = v;
		return ;
	}
	else
	{
		int mid = (l+r) >> 1;
		if(ind <= mid)
			update(p<<1, l, mid, ind, v);
		else
			update(p<<1|1, mid+1, r, ind, v);
		tree[p] = max(tree[p<<1], tree[p<<1|1]);
	}
}

ll query(int p, int l, int r, int x, int y)
{
	if(x <= l && r <= y)
	{
		return tree[p];
	}
	else
	{
		int mid = (l+r) >> 1;
		ll ans = -1;
		if(x <= mid)
			ans = max(ans, query(p<<1, l, mid, x, y));
		if(y > mid)
			ans = max(ans, query(p<<1|1, mid+1, r, x, y));
		return ans;
	}
}

int main()
{
	while(scanf("%d %d", &n, &m) != EOF)
	{
		for(int i=1;i<=n;i++)
		{
			scanf("%d", &num[i]);
		}
		build(1, 1, n);
		for(int i=1;i<=m;i++)
		{
			char a;
			int x, y;
			scanf(" %c%d%d", &a, &x, &y);
			if(a == 'Q')
			{
				printf("%lld\n", query(1, 1, n, x, y));
			}
			else
			{
				update(1, 1, n, x, y);
			}
		}
	}
	return 0;
}
\end{lstlisting}


\subsubsection{区间染色+统计+离散化 POJ2528}
\begin{lstlisting}
#include<iostream>
#include<algorithm>
#include<string.h>
using namespace std;
typedef long long ll;

const int maxn = 20000 + 100;
int tree[maxn<<4];
int li[maxn],ri[maxn];
bool vis[maxn];
int lisan[maxn*3];
int ans = 0;

void init()
{
	memset(tree, -1, sizeof(tree));
	memset(vis, 0, sizeof(vis));
	ans = 0;
}

void pushdown(int p)
{
	tree[p<<1] = tree[p<<1|1] = tree[p];
	tree[p] = -1;
}

void update(int p, int l, int r, int x, int y, int v)
{
	if(x <= l && r <= y)
	{
		tree[p] = v;
		return ;
	}
	if(tree[p]!=-1)
		pushdown(p);
	int mid = (l+r)>>1;
	if(x <= mid)
		update(p<<1, l, mid, x, y, v);
	if(y > mid)
		update(p<<1|1, mid+1, r, x, y, v);
	tree[p] = -1;
}

void query(int p, int l, int r)
{
	if(tree[p]!=-1)
	{
		if(vis[tree[p]]==0)
		{
			vis[tree[p]] = 1;
			ans++;
		}
		return;
	}
	if(l==r)return;
	int mid = (l+r)>>1;
	query(p<<1, l, mid);
	query(p<<1|1, mid+1, r);
}

int main()
{
	int t;
	cin>>t;
	while(t--)
	{
		init();
		int n;
		cin>>n;
		int tot = 0;
		for(int i=0;i<n;i++)
		{
			cin>>li[i]>>ri[i];
			lisan[tot++] = li[i];
			lisan[tot++] = ri[i];
		}
		sort(lisan, lisan+tot);
		int m = unique(lisan, lisan+tot) - lisan;
		int t = m;
		for(int i=1;i<t;i++)
		{
			if(lisan[i]-lisan[i-1]>1)
			{
				lisan[m++] = lisan[i-1]+1;
			}
		}
		sort(lisan, lisan+m);
		for(int i=0;i<n;i++)
		{
			int x,y;
			x = lower_bound(lisan, lisan+m, li[i]) - lisan;
			y = lower_bound(lisan, lisan+m, ri[i]) - lisan;
			update(1, 0, m-1, x, y, i);
		}
		query(1, 0, m-1);
		cout<<ans<<endl;
	}
	return 0;
}
\end{lstlisting}

\subsubsection{线段树+扫描线求矩阵覆盖周长 POJ1177}
\begin{lstlisting}
#include<iostream>
#include<vector>
#include<algorithm>
#include<cmath>
using namespace std;
typedef long long ll;

const int maxn = 10005;
vector<int> x;
int getID(int v)
{
	return lower_bound(x.begin(), x.end(), v) - x.begin();
}

struct Segment
{
	int l, r;
	int h;
	int flag;
}segment[maxn];
bool cmp(Segment a, Segment b)
{
	return a.h < b.h;
}

struct Node
{
	int l,r;
	int lr, rr;
	int len;
	int line;
	int s;
}node[maxn<<2];

void build(int p, int l, int r)
{
	node[p].l = l; node[p].r = r;
	node[p].line = node[p].len = node[p].s = 0;
	node[p].lr = node[p].rr = 0;
	if(l==r)return;
	int mid = (l+r)>>1;
	build(p<<1, l, mid);
	build(p<<1|1, mid+1, r);
}

void pushup(int p)
{
	if(node[p].s)
	{
		node[p].line = 1;
		node[p].rr = node[p].lr = 1;
		node[p].len = x[node[p].r+1] - x[node[p].l];
		return;
	}
	else if(node[p].l == node[p].r)
	{
		node[p].lr = node[p].rr = node[p].line = node[p].len = 0;
	}
	else
	{
		node[p].lr = node[p<<1].lr;
		node[p].rr = node[p<<1|1].rr;
		node[p].len = node[p<<1].len + node[p<<1|1].len;
		node[p].line = node[p<<1].line + node[p<<1|1].line - (node[p<<1].rr&&node[p<<1|1].lr);
	}
}

void update(int p, int l, int r, int v)
{
	if(node[p].r < l || node[p].l > r)return;
	if(l <= node[p].l && node[p].r <= r)
	{
		node[p].s += v;
		pushup(p);
		return;
	}
	update(p<<1, l, r, v);
	update(p<<1|1, l, r, v);
	pushup(p);
}

int main()
{
	int n;
	cin>>n;
	for(int i=1;i<=n;i++)
	{
		int x1,x2,y1,y2;
		cin>>x1>>y1>>x2>>y2;
		Segment &s1 = segment[2*i-1];
		Segment &s2 = segment[i<<1];
		s1.l = s2.l = x1;
		s1.r = s2.r = x2;
		s1.h = y1; s2.h = y2;
		s1.flag = 1; s2.flag = -1;
		x.push_back(x1);
		x.push_back(x2);
	}
	sort(segment+1, segment+2*n+1, cmp);

	sort(x.begin(), x.end());
	x.erase(unique(x.begin(), x.end()), x.end());

	build(1, 0, x.size()-1);

	ll ans = 0;
	int last = 0;
	for(int i=1;i<=2*n;i++)
	{
		int l = getID(segment[i].l);
		int r = getID(segment[i].r);
		update(1, l, r-1, segment[i].flag);
		ans += abs(node[1].len - last);
		if(i!=2*n)
			ans += node[1].line * 2 * (segment[i+1].h - segment[i].h);
		last = node[1].len;
	}
	cout<<ans<<endl;
	return 0;
}
\end{lstlisting}

\subsubsection{线段树+扫描线求矩阵面积并 HDU1542}
\begin{lstlisting}
#include<bits/stdc++.h>
using namespace std;
typedef long long ll;

const int maxn = 210;
int n;
vector<double> x;
inline int getID(double v)
{
	return lower_bound(x.begin(), x.end(), v) - x.begin();
}

struct Segment
{
	double l, r;
	double h;
	int flag;
}segment[maxn];
bool cmp(Segment a, Segment b)
{
	return a.h < b.h;
}

struct Node
{
	int l, r;
	int s;
	double len;
}node[maxn<<2];

void pushup(int p)
{
	if(node[p].s)
		node[p].len = x[node[p].r+1] - x[node[p].l];
	else if(node[p].l == node[p].r)
		node[p].len = 0;
	else
		node[p].len = node[p<<1].len + node[p<<1|1].len;
}

void build(int p, int l, int r)
{
	if(l>r)return;
	node[p].l = l; node[p].r = r;
	node[p].s = 0; node[p].len = 0;
	if(l==r) return;
	int mid = (l+r)>>1;
	build(p<<1, l, mid);
	build(p<<1|1, mid+1, r);
	pushup(p);
}

void update(int p, int l, int r, int v)
{
	if(l>node[p].r || r<node[p].l) return;
	if(l <= node[p].l && node[p].r <= r)
	{
		node[p].s += v;
		pushup(p);
		return;
	}
	update(p<<1, l, r, v);
	update(p<<1|1, l, r, v);
	pushup(p);
}

int main()
{
	int cas = 0;
	while(scanf("%d", &n) && n)
	{
		x.clear();
		for(int i=1;i<=n;i++)
		{
			double x1,x2,y1,y2;
			scanf("%lf%lf%lf%lf", &x1, &y1, &x2, &y2);
			Segment &s1 = segment[2*i-1];
			Segment &s2 = segment[i<<1];
			s1.l=s2.l=x1;
			s1.r=s2.r=x2;
			s1.h=y1;
			s2.h=y2;
			s1.flag=1;
			s2.flag=-1;
			x.push_back(x1);
			x.push_back(x2);
		}
		sort(segment+1, segment+2*n+1, cmp);

		sort(x.begin(), x.end());
		x.erase(unique(x.begin(), x.end()), x.end());

		build(1, 0, x.size()-1);
		double ans = 0;
		for(int i=1;i<=2*n;i++)
		{
			int l=getID(segment[i].l);
			int r=getID(segment[i].r);
			update(1, l, r-1, segment[i].flag);
			ans+=node[1].len*(segment[i+1].h - segment[i].h);
		}
		printf("Test case #%d\n", ++cas);
		printf("Total explored area: %.2f\n\n", ans);
	}
	return 0;
}
\end{lstlisting}


\subsection{主席树}
\subsubsection{区间第K大}
\begin{lstlisting}
#include<bits/stdc++.h>
using namespace std;
typedef long long ll;
#define inf 0x3f3f3f3f
typedef pair<int, int> P;
const int maxn = 1e6+5;
const ll mod = 1e9+7;

// 顶点,代表区间[l, r]中有v个数字
struct node
{
	int l, r, v;
}tree[maxn*20]; // 空间开大一点,因为要动态开点

// edit[i]存的是第i颗权值线段树的根节点在tree数组中的位置
// a是存放原数据的数组, b是离散化后的数组, tot表示顶点的个数
int edit[maxn], a[maxn], b[maxn], tot=0;

// 建树
int build(int l, int r)
{
	// 这里就是动态开辟新的结点,就是将tot加一
	int pos = ++tot;
	tree[pos].v = 0; //初始化为0
	if(l==r)return pos; //到根节点了,返回
	// 二分建树没什么好说的
	int mid = (l+r)>>1;
	tree[pos].l = build(l, mid);
	tree[pos].r = build(mid+1, r);
	//要返回pos的位置,因为edit数组要存新根的位置
	return pos;
}

// 就是插入操作,插入新的元素
// ed是前一版本的结点在tree的位置, 更新区间[l, r], 位置为v
int update(int ed, int l, int r, int v)
{
	// 动态开点
	int pos = ++tot;
	//先将新一版本的当前结点复制为上一个版本的对应结点
	tree[pos] = tree[ed]; tree[pos].v++; //新一版本的结点的v要加一,因为对应区间插入了一个数
	if(l==r) return pos; //到叶子节点了,返回
	// 二分
	int mid = (l+r)>>1;
	// 如果更新位置v在左子树中,递归更新即可,在右子树中同理,最后要返回pos
	if(v<=mid) tree[pos].l = update(tree[ed].l, l, mid, v);
	else tree[pos].r = update(tree[ed].r, mid+1, r, v);
	return pos;
}

// 查询区间[l, r]第k大的数字,pre对应l-1版本的权值线段树的节点位置, ed代表r版本的权值线段树的节点位置
int query(int pre, int ed, int l, int r, int k)
{
	// 到叶子节点就返回
	if(l==r) return l;
	// 二分
	int mid = (l+r)>>1;
	// 先计算左子树的数字个数
	int x = tree[tree[ed].l].v - tree[tree[pre].l].v;
	// 若左子树的数字个数大于等于k,说明我们要找的数字在左子树中,递归走到左子树继续寻找
	if(x>=k) return query(tree[pre].l, tree[ed].l, l, mid, k);
	// 否则就在右子树中,我们要在右子树中寻找k-x大的数字,递归寻找就好
	else return query(tree[pre].r, tree[ed].r, mid+1, r, k-x);
}

int main()
{
	int n, q; // n个数字, q次询问
	scanf("%d%d", &n, &q); // 这一题cin/cout会被卡
	// 输入数据,并copy到b数组中
	for(int i=1;i<=n;i++)
	{
		scanf("%d", &a[i]);
		b[i] = a[i];
	}
	// 离散化
	sort(b+1, b+1+n);
	int m = unique(b+1, b+n+1) - b-1;
	// 构建一颗空的权值线段树,edit[0]存放的就是这颗空树的根节点的位置
	edit[0] = build(1, m);
	// 插入n个数据
	for(int i=1;i<=m;i++)
	{
		// 找到a[i]离散化后对应的位置
		a[i] = lower_bound(b+1, b+m+1, a[i]) - b;
		// edit[i]存放第i版本的权值线段树的根节点位置
		edit[i] = update(edit[i-1], 1, m, a[i]);
	}
	// 处理q次询问
	while(q--)
	{
		int x, y, k;
		scanf("%d%d%d", &x, &y, &k);
		// pos对应的是离散化后的位置,所以最后输出b[pos]即可
		int pos = query(edit[x-1], edit[y], 1, m, k);
		printf("%d\n", b[pos]);
	}
	return 0;
}
\end{lstlisting}


\subsubsection{动态区间第K大(主席树套树状数组) ZOJ2112}
\begin{lstlisting}
#include<bits/stdc++.h>
using namespace std;
typedef long long ll;
#define inf 0x3f3f3f3f
typedef pair<int, int> P;
const int maxn = 60010;
const ll mod = 1e9+7;
const int M = 2500010;

int n, m, q, tot;
struct node
{
	int l, r, v;
}tree[M];

// T是主席树,与上面代码的edit作用一样, S数组就是树状数组, use数组是树状数组求和时用的,记录的是树状数组中哪一些权值线段树要被用来求和
int T[maxn], S[maxn], use[maxn], a[maxn], b[maxn];

// 记录询问,因为要将修改后的值一起构建主席树,所以将在线转为离线
struct Q
{
	// 对于查询区间[l, r]第k大的询问来说,flag为1
	//若是修改操作,l记录修改的位置,r记录新值,flag为0
	int l, r, k, flag;
}query[10010];

// 快速找出x在离散化后的位置
int HASH(int x)
{
	return lower_bound(b+1, b+m+1, x) - b;
}

// 建静态主席树,和之前的一样
int build(int l, int r)
{
	int pos = ++tot;
	tree[pos].v = 0;
	if(l==r) return pos;
	int mid = (l+r)>>1;
	tree[pos].l = build(l, mid);
	tree[pos].r = build(mid+1, r);
	return pos;
}

// 和之前的静态主席树update差不多,只不过不是直接将tree[pos].v+1,而是加参数v
//消除影响v就为-1, 添加影响v就为1, 其他没什么不同
int update(int ed, int l, int r, int p, int v)
{
	int pos = ++tot;
	tree[pos] = tree[ed];
	tree[pos].v += v;
	if(l==r) return pos;
	int mid = (l+r)>>1;
	if(p<=mid) tree[pos].l = update(tree[ed].l, l, mid, p, v);
	else tree[pos].r = update(tree[ed].r, mid+1, r, p, v);
	return pos;
}

// 树状数组的lowbit
int lowbit(int x) { return x&(-x); }

// 修改操作,修改位置x的影响
int add(int x, int v)
{
	// 找出a[x]在离散化后的位置p
	int p = HASH(a[x]);
	while(x<=n)
	{
		// 修改操作,对树状数组中相应的权值线段树进行修改,消除影响:v=-1, 添加影响:v=1
		// 因为树状数组中的权值线段树不需要可持久化,所以直接在原版本上修改就可以了
		S[x] = update(S[x], 1, m, p, v);
		x+=lowbit(x);
	}
}

// 树状数组求和, 求左子树包含的数字个数,和静态主席树一样的思想,都是先求左子树
int sum(int x)
{
	int ret = 0;
	while(x)
	{
		// use[i]记录的就是树状数组中相应的权值线段树的结点位置
		// 似乎这一句有一点点难以理解,结合整体代码多看几遍吧
		ret += tree[tree[use[x]].l].v;
		x-=lowbit(x);
	}
	return ret;
}

// 询问操作。树状数组求[pre, ed]的和, tpre和ted是静态主席树的区间左右顶点的位置, 区间[l, r]第k大
int Query(int pre, int ed, int tpre, int ted, int l, int r, int k)
{
	if(l==r)return l;
	int mid = (l+r)>>1;
	// sum就是树状数组的求和,相对于静态主席树,多了sum求修改操作的影响
	// tmp为当前左子树的数字个数
	int tmp = sum(ed) - sum(pre) + tree[tree[ted].l].v - tree[tree[tpre].l].v;
	// 若左子树的数字个数大于等于k就往左子树继续走,这里和主席树没什么区别
	if(tmp >= k)
	{
		// 两个for循环是更新左子树需要用到的树状数组中的权值线段树的位置,有那么一丢丢难以理解,多看几遍?
		for(int i=ed; i; i-=lowbit(i)) use[i] = tree[use[i]].l;
		for(int i=pre; i; i-=lowbit(i)) use[i] = tree[use[i]].l;
		return Query(pre, ed, tree[tpre].l, tree[ted].l, l, mid, k);
	}
	else
	{
		// 走右子树同理
		for(int i=ed; i; i-=lowbit(i)) use[i] = tree[use[i]].r;
		for(int i=pre; i; i-=lowbit(i)) use[i] = tree[use[i]].r;
		return Query(pre, ed, tree[tpre].r, tree[ted].r, mid+1, r, k-tmp);
	}
}

int main()
{
	int t; // t个case
	scanf("%d", &t);
	while(t--)
	{
		scanf("%d%d", &n, &q);
		m = tot = 0; // 记得初始化
		for(int i=1;i<=n;i++)
		{
			scanf("%d", &a[i]);
			b[++m] = a[i];
		}
		char op[5];
		for(int i=1;i<=q;i++)
		{
			scanf("%s", op);
			// 查询操作
			if(op[0]=='Q')
			{
				scanf("%d%d%d", &query[i].l, &query[i].r, &query[i].k);
				query[i].flag = 1;
			}
			else
			{
				scanf("%d%d", &query[i].l, &query[i].r);
				b[++m] = query[i].r; // 注意要将修改后的新值加入到待离散化的b数组
				query[i].flag = 0;
			}
		}
		// 离散化
		sort(b+1, b+m+1);
		m = unique(b+1, b+m+1) - b-1;
		// 构建主席树
		T[0] = build(1, m);
		for(int i=1;i<=n;i++)
			T[i] = update(T[i-1], 1, m, HASH(a[i]), 1);
		// 构建树状数组,每一个节点都是一颗空的权值线段树
		for(int i=1;i<=n;i++)
			S[i] = T[0];
		// 离线处理q个询问
		for(int i=1;i<=q;i++)
		{
			if(query[i].flag) // 查询
			{
				// 两个for循环标记区间[l, r]要使用的树状数组中的权值线段树的位置
				for(int j=query[i].r; j; j-=lowbit(j)) use[j] = S[j];
				for(int j=query[i].l-1; j; j-=lowbit(j)) use[j] = S[j];
				printf("%d\n", b[Query(query[i].l-1, query[i].r, T[query[i].l-1], T[query[i].r], 1, m, query[i].k)]);
			}
			else
			{
				// 先消除影响
				add(query[i].l, -1);
				// 在原数组中更新值
				a[query[i].l] = query[i].r;
				//添加新值的影响
				add(query[i].l, 1);
			}
		}
	}
	return 0;
}
\end{lstlisting}







\section{DP}
\subsection{背包}
\begin{lstlisting}
const int MAXN = 10000;
const int SIZE = 100000;

int dp[SIZE];
int volume[MAXN], value[MAXN], c[MAXN];
int n, v;           //  总物品数,背包容量

//  01背包
void ZeroOnepark(int val, int vol)
{
	for (int j = v ; j >= vol; j--)
	{
		dp[j] = max(dp[j], dp[j - vol] + val);
	}
}

//  完全背包
void Completepark(int val, int vol)
{
	for (int j = vol; j <= v; j++)
	{
		dp[j] = max(dp[j], dp[j - vol] + val);
	}
}

//  多重背包
void Multiplepark(int val, int vol, int amount)
{
	if (vol * amount >= v)
	{
		Completepark(val, vol);
	}
	else
	{
		int k = 1;
		while (k < amount)
		{
			ZeroOnepark(k * val, k * vol);
			amount -= k;
			k <<= 1;
		}
		if (amount > 0)
		{
			ZeroOnepark(amount * val, amount * vol);
		}
	}
}

int main()
{
	while (cin >> n >> v)
	{
		for (int i = 1 ; i <= n ; i++)
		{
			cin >> volume[i] >> value[i] >> c[i];      //   费用,价值,数量
		}
		memset(dp, 0, sizeof(dp));
		for (int i = 1; i <= n; i++)
		{
			Multiplepark(value[i], volume[i], c[i]);  
		}
		cout << dp[v] << endl;
	}
	return 0;
}
\end{lstlisting}


\subsection{最长公共递增子序列}
\begin{lstlisting}
/*
*  最长公共递增子序列 O(n^2)
*  f记录路径,DP记录长度, 用a对b扫描,逐步最优化。
*/
const int N = 1010;

int f[N][N], dp[N];

int gcis(int a[], int la, int b[], int lb, int ans[])
{   //  a[1...la], b[1...lb]
	int i, j, k, mx;
	memset(f, 0, sizeof(f));
	memset(dp, 0, sizeof(dp));
	for (i = 1; i <= la; i++)
	{
		memcpy(f[i], f[i-1], sizeof(f[0]));
		for (k = 0, j = 1; j <= lb; j++)
		{
			if (b[j - 1] < a[i - 1] && dp[j] > dp[k])
			{
				k = j;
			}
			if (b[j - 1] == a[i - 1] && dp[k] + 1 > dp[j])
			{
				dp[j] = dp[k] + 1,
				f[i][j] = i * (lb + 1) + k;
			}
		}
	}
	for (mx = 0, i = 1; i <= lb; i++)
	{
		if (dp[i] > dp[mx])
		{
			mx = i;
		}
	}
	for (i = la * lb + la + mx, j = dp[mx]; j; i = f[i / (lb + 1)][i % (lb + 1)], j--)
	{
		ans[j - 1] = b[i % (lb + 1) - 1];
	}
	return dp[mx];
}
\end{lstlisting}

\subsection{最长公共子序列}
\begin{lstlisting}
const int N = 1010;

int a[N][N];

int LCS(const char *s1, const char *s2)
{   //  s1:0...m, s2:0...n
	int m = (int)strlen(s1), n = (int)strlen(s2);
	int i, j;
	a[0][0] = 0;
	for (i = 1; i <= m; ++i)
	{
		a[i][0] = 0;
	}
	for (i = 1; i <= n; ++i)
	{
		a[0][i] = 0;
	}
	for (i = 1; i <= m; ++i)
	{
		for (j = 1; j <= n; ++j)
		{
			if (s1[i - 1] == s2[j - 1])
			{
				a[i][j] = a[i - 1][j - 1] + 1;
			}
			else if (a[i - 1][j] > a[i][j - 1])
			{
				a[i][j]= a[i - 1][j];
			}
			else
			{
				a[i][j] = a[i][j - 1];
			}
		}
	}
	return a[m][n];
}
\end{lstlisting}

\subsection{最长有序子序列}
\begin{lstlisting}
/*
*  递增(默认)
*  递减 
*  非递增
*  非递减 (1)>= && <  (2)<  (3)>=
*/
const int MAXN = 1001;

int a[MAXN], f[MAXN], d[MAXN];   //  d[i] 用于记录 a[0...i] 以 a[i] 结尾的最大长度

int bsearch(const int *f, int size, const int &a)
{
	int l = 0, r = size - 1;
	while (l <= r)
	{
		int mid = (l + r) / 2;
		if (a > f[mid - 1] && a <= f[mid])  //  (1)
		{
			return mid; 
		}
		else if (a < f[mid])
		{
			r = mid - 1;
		}
		else
		{
			l = mid + 1;
		}
	}
	return -1;
}

int LIS(const int *a, const int &n)
{
	int i, j, size = 1;
	f[0] = a[0];
	d[0] = 1;
	for (i = 1; i < n; ++i)
	{
		if (a[i] <= f[0])               //  (2)
		{
			j = 0;
		}
		else if (a[i] > f[size - 1])    //  (3)
		{
			j = size++;
		}
		else
		{
			j = bsearch(f, size, a[i]);
		}
		f[j] = a[i];
		d[i] = j + 1;
	}
	return size;
}

int main()
{
	int i, n;
	while (scanf("%d", &n) != EOF)
	{
		for (i = 0; i < n; ++i)
		{
			scanf("%d", &a[i]);
		}
		printf("%d\n", LIS(a, n));      // 求最大递增 / 上升子序列(如果为最大非降子序列,只需把上面的注释部分给与替换)
	}
	return 0;
}
\end{lstlisting}





\section{字符串}


\section{数学}
\subsection{快速幂}
\begin{lstlisting}
typedef long long ll;
ll pow(ll x,ll n,ll mod)
{
	ll res=1;
	while(n>0)
	{
		if(n&1)	
		{
			res=res*x;
			res=res%mod;
		}
		x=x*x;
		x=x%mod;
		n>>=1;
	}
	return res;	
}
\end{lstlisting}

\subsection{逆序数}
\begin{lstlisting}
/*
*  也可以用树状数组做
*  a[0...n-1] cnt=0; call: MergeSort(0, n)
*/
const int N = 1010;
int a[N];
int c[N];
int cnt = 0;

void MergeSort(int l, int r)
{
	int mid, i, j, tmp;
	if (r > l + 1)
	{
		mid = (l + r) / 2;
		MergeSort(l, mid);
		MergeSort(mid, r);
		tmp = l;
		for (i = l, j = mid; i < mid && j < r;)
		{
			if (a[i] > a[j])
			{
				c[tmp++] = a[j++];
				cnt += mid - i;
			}
			else
			{
				c[tmp++] = a[i++];
			}
		}
		if (j < r)
		{
			for (; j < r; ++j)
			{
				c[tmp++] = a[j];
			}
		}
		else
		{
			for (; i < mid; ++i)
			{
				c[tmp++]=a[i];
			}
		}
		for (i = l; i < r; ++i)
		{
			a[i] = c[i];
		}
	}
	return ;
}
\end{lstlisting}

\subsection{无序序列变有序的最少交换次数}
\subsubsection{相邻元素交换}
\begin{lstlisting}
\\ 等于逆序数个数
\end{lstlisting}

\subsubsection{任意元素交换}
\begin{lstlisting}
*
*  交换任意两数的本质是改变了元素位置,
*  故建立元素与其目标状态应放置位置的映射关系
*/
int getMinSwaps(vector<int> &A)
{
	//  排序
	vector<int> B(A);
	sort(B.begin(), B.end());
	map<int, int> m;
	int len = (int)A.size();
	for (int i = 0; i < len; i++)
	{
		m[B[i]] = i;    //  建立每个元素与其应放位置的映射关系
	}

	int loops = 0;      //  循环节个数
	vector<bool> flag(len, false);
	//  找出循环节的个数
	for (int i = 0; i < len; i++)
	{
		if (!flag[i])
		{
			int j = i;
			while (!flag[j])
			{
				flag[j] = true;
				j = m[A[j]];    //  原序列中j位置的元素在有序序列中的位置
			}
			loops++;
		}
	}
	return len - loops;
}
\end{lstlisting}

\subsection{GCD}
\subsubsection{非递归}
\begin{lstlisting}
int gcd(int x, int y)
{
	if (!x || !y)
	{
		return x > y ? x : y;
	}

	for (int t; t = x % y, t; x = y, y = t) ;

	return y;
}
\end{lstlisting}

\subsubsection{递归}
\begin{lstlisting}
int gcd(int a, int b)
{ 
	return b==0 ? a : gcd(b, a%b); 
}
\end{lstlisting}

\subsection{EXGCD}
\begin{lstlisting}
/*
*  求x,y使得gcd(a, b) = a * x + b * y;
*/
int extgcd(int a, int b, int &x, int &y)
{
	if (b == 0)
	{
		x = 1;
		y = 0;
		return a;
	}

	int d = extgcd(b, a % b, x, y);
	int t = x;
	x = y;
	y = t - a / b * y;

	return d;
}
\end{lstlisting}

\subsection{素数}
\subsubsection{判断小于MAXN的数是否为素数}
\begin{lstlisting}
/*
*  素数筛选,判断小于MAXN的数是不是素数
*  notprime是一张表,false表示是素数,true表示不是
*/
const int MAXN = 1000010;
bool notprime[MAXN];

void init()
{
	memset(notprime, false, sizeof(notprime));
	notprime[0] = notprime[1] = true;
	for (int i = 2; i < MAXN; i++)
	{
		if (!notprime[i])
		{
			if (i > MAXN / i)   //  阻止后边i * i溢出(或者i,j用long long)
			{
				continue;
			}
			//  直接从i * i开始就可以,小于i倍的已经筛选过了
			for (int j = i * i; j < MAXN; j += i)
			{
				notprime[j] = true;
			}
		}
	}
}
\end{lstlisting}

\subsubsection{生成小于等于MAXN的全部素数}
\begin{lstlisting}
/*
*  素数筛选,查找出小于等于MAXN的素数
*  prime[0]存素数的个数
*/

const int MAXN = 100000;
int prime[MAXN + 1];

void getPrime()
{
	memset(prime, 0, sizeof(prime));
	for (int i = 2; i <= MAXN; i++)
	{
		if (!prime[i])
		{
			prime[++prime[0]] = i;
		}
		for (int j = 1; j <= prime[0] && prime[j] <= MAXN / i; j++)
		{
			prime[prime[j] * i] = 1;
			if (i % prime[j] == 0)
			{
				break;
			}
		}
	}
}
\end{lstlisting}

\subsubsection{Miller Rabin}
\begin{lstlisting}
/*
*  随机素数测试(伪素数原理)
*  CALL: bool res = miller(n);
*  快速测试n是否满足素数的“必要”条件,出错概率极低
*  对于任意奇数n > 2和正整数s,算法出错概率≤2^(-s)
*/

int witness(int a, int n)
{
	int x, d = 1;
	int i = ceil(log(n - 1.0) / log(2.0)) - 1;
	for (; i >= 0; i--)
	{
		x = d;
		d = (d * d) % n;
		if (d == 1 && x != 1 && x != n - 1)
		{
			return 1;
		}
		if (((n - 1) & (1 << i)) > 0)
		{
			d = (d * a) % n;
		}
	}
	return (d == 1 ? 0 : 1);
}

int miller(int n, int s = 50)
{
	if (n == 2)     //  质数返回1
		return 1;
	if (n % 2 == 0) //  偶数返回0
		return 0;
	int j, a;
	for (j = 0; j < a; j++)
	{
		a = rand() * (n - 2) / RAND_MAX + 1;
		//  rand()只能随机产生[0, RAND_MAX)内的整数
		//  而且这个RAND_MAX只有32768直接%n的话是永远
		//  也产生不了[RAND_MAX, n)之间的数
		if (witness(a, n))
		{
			return 0;
		}
	}
	return 1;
}
\end{lstlisting}

\subsection{组合数}
\begin{lstlisting}
int com(int n, int r)   //  return C(n, r)
{
	if (n - r > r)
	{
		r = n - r;      //  C(n, r) = C(n, n - r)
	}
	int i, j, s = 1;
	for (i = 0, j = 1; i < r; ++i)
	{
		s *= (n - i);
		for (; j <= r && s % j == 0; ++j)
		{
			s /= j;
		}
	}
	return s;
}
\end{lstlisting}

\subsection{阶乘长度}
\begin{lstlisting}
#define PI 3.1415926

int main()
{
	int n, a;
	while (~scanf(“%d", &n))
	{
		a = (int)((0.5 * log(2 * PI * n) + n * log(n) - n) / log(10));
		printf("%d\n", a + 1);
	}
	return 0;
}
\end{lstlisting}

\subsection{全排列}
\begin{lstlisting}
#define MAX_N 10
int n;                      //  共n个数
int rcd[MAX_N];             //  记录每个位置填的数
int used[MAX_N];            //  标记数是否用过
int num[MAX_N];             //  存放输入的n个数

void full_permutation(int l)
{
	int i;
	if (l == n)
	{
		for (i = 0; i < n; i++)
		{
			printf("%d", rcd[i]);
			if (i < n-1)
			{
				printf(" ");
			}
		}
		printf("\n");
		return ;
	}
	for (i = 0; i < n; i++)         //  枚举所有的数(n个),循环从开始
	if (!used[i])
	{                           //  若num[i]没有使用过, 则标记为已使用
		used[i] = 1;
		rcd[l] = num[i];        //  在l位置放上该数
		full_permutation(l+1);  //  填下一个位置
		used[i] = 0;            //  清标记
	}
}
\end{lstlisting}

\subsection{求斐波那契第N项}
\begin{lstlisting}
/*
*  求斐波那契数列第N项,模MOD
*/
#define mod(a, m) ((a) % (m) + (m)) % (m)
const int MOD = 1e9 + 9;
struct MATRIX
{
	long long a[2][2];
};

MATRIX a;
long long f[2];

void ANS_Cf(MATRIX a)
{
	f[0] = mod(a.a[0][0] + a.a[1][0], MOD);
	f[1] = mod(a.a[0][1] + a.a[1][1], MOD);
	return ;
}

MATRIX MATRIX_Cf(MATRIX a, MATRIX b)
{
	MATRIX ans;
	int k;
	for (int i = 0; i < 2; i++)
	{
		for (int j = 0; j < 2; j++)
		{
			ans.a[i][j] = 0;
			k = 0;
			while (k < 2)
			{
				ans.a[i][j] += a.a[k][i] * b.a[j][k];
				ans.a[i][j] = mod(ans.a[i][j], MOD);
				++k;
			}
		}
	}
	return ans;
}

MATRIX MATRIX_Pow(MATRIX a, long long n)
{
	MATRIX ans;
	ans.a[0][0] = 1;
	ans.a[1][1] = 1;
	ans.a[0][1] = 0;
	ans.a[1][0] = 0;
	while (n)
	{
		if (n & 1)
		{
			ans = MATRIX_Cf(ans, a);
		}
		n = n >> 1;
		a = MATRIX_Cf(a, a);
	}
	return ans;
}

int main()
{
	long long n;
	while (cin >> n)
	{
		if (n == 1)
		{
			cout << '1' << '\n';
			continue;
		}
		a.a[0][0] = a.a[0][1] = a.a[1][0] = 1;
		a.a[1][1] = 0;
		a = MATRIX_Pow(a, n - 2);
		ANS_Cf(a);
		cout << f[0] << '\n';
	}
	return 0;
}
\end{lstlisting}



\section{STL}
\subsection{vector}
\begin{lstlisting}
vector<int> s;      
//  定义一个空的vector对象,存储的是int类型的元素
vector<int> s(n);   
//  定义一个含有n个int元素的vector对象
vector<int> s(first, last); 
//  定义一个vector对象,并从由迭代器first和last定义的序列[first, last)中复制初值

s[i]                //  直接以下标方式访问容器中的元素
s.front()           //  返回首元素
s.back()            //  返回尾元素
s.push_back(x)      //  向表尾插入元素x
s.size()            //  返回表长
s.empty()           //  表为空时,返回真,否则返回假
s.pop_back()        //  删除表尾元素
s.begin()           //  返回指向首元素的随机存取迭代器
s.end()             //  返回指向尾元素的下一个位置的随机存取迭代器
s.insert(it, val)   //  向迭代器it指向的元素前插入新元素val
s.insert(it, n, val)//  向迭代器it指向的元素前插入n个新元素val
s.insert(it, first, last)   
//  将由迭代器first和last所指定的序列[first, last)插入到迭代器it指向的元素前面
s.erase(it)         //  删除由迭代器it所指向的元素
s.erase(first, last)//  删除由迭代器first和last所指定的序列[first, last)
s.reserve(n)        //  预分配缓冲空间,使存储空间至少可容纳n个元素
s.resize(n)         //  改变序列长度,超出的元素将会全部被删除,如果序列需要扩展(原空间小于n),元素默认值将填满扩展出的空间
s.resize(n, val)    //	改变序列长度,超出的元素将会全部被删除,如果序列需要扩展(原空间小于n),val将填满扩展出的空间
s.clear()           //  删除容器中的所有元素
s.swap(v)           //  将s与另一个vector对象进行交换
s.assign(first, last)	//  将序列替换成由迭代器first和last所指定的序列[first, last),[first, last)不能是原序列中的一部分

//  要注意的是,resize操作和clear操作都是对表的有效元素进行的操作,但并不一定会改变缓冲空间的大小
//  vector上还定义了序列之间的比较操作运算符(>、<、>=、<=、==、!=),可以按照字典序比较两个序列。
\end{lstlisting}


\subsection{set}
\begin{lstlisting}
// 有序
set<int> s;
s.begin()       //  返回指向第一个元素的迭代器
s.clear()       //  清除所有元素
s.count()       //  返回某个值元素的个数
s.empty()       //  如果集合为空,返回true(真)
s.end()         //  返回指向最后一个元素之后的迭代器,不是最后一个元素
s.equal_range() //  返回集合中与给定值相等的上下限的两个迭代器
s.erase()       //  删除集合中的元素
s.find()        //  返回一个指向被查找到元素的迭代器
s.get_allocator()   //  返回集合的分配器
s.insert()      //  在集合中插入元素
s.lower_bound() //  返回指向大于(或等于)某值的第一个元素的迭代器
s.key_comp()    //  返回一个用于元素间值比较的函数
s.max_size()    //  返回集合能容纳的元素的最大限值
s.rbegin()      //  返回指向集合中最后一个元素的反向迭代器
s.rend()        //  返回指向集合中第一个元素的反向迭代器
s.size()        //  集合中元素的数目
s.swap()        //  交换两个集合变量
s.upper_bound() //  返回大于某个值元素的迭代器
s.value_comp()  //  返回一个用于比较元素间的值的函数

// 多重集合
multiset<int> s;
// 操作类似
\end{lstlisting}


\subsection{pair}
\begin{lstlisting}
pair<T1, T2> p1;
pair<T1, T2> p1(v1, v2);
p1.first;
p1.second;
p1 = make_pair(v1, v2);

vector<pair<int, int> > v;
sort(v.begin(), v.end()); //根据pair的first排序,从小到大
\end{lstlisting}


\subsection{stack}
\begin{lstlisting}
stack<int> s;
s.push(x);  //  入栈
s.pop();    //  出栈
s.top();    //  访问栈顶
s.empty();  //  当栈空时,返回true
s.size();   //  访问栈中元素个数
\end{lstlisting}


\subsection{queue}
\begin{lstlisting}
queue<int> q;
q.push(x);  //  入队列
q.pop();    //  出队列
q.front();  //  访问队首元素
q.back();   //  访问队尾元素
q.empty();  //  判断队列是否为空
q.size();   //  访问队列中的元素个数

//priority_queue(优先队列)。优先队列与队列的差别在于优先队列不是按照入队的顺序出队,而是按照队列中元素的优先权出队列(默认为大者优先,也可以通过指定算子来指定自己的优先顺序)。

priority_queue模版类有三个模版参数,第一个是元素类型,第二个是容器类型,第三个是比较算子。其中后两者都可以忽略,默认容器为vector,默认算子为less,即小的往前排,大的往后排(出队列时列尾元素先出队)。

priority_queue<int> q;
priority_queue<pair<int, int> > qq;	//  注意在两个尖括号之间一定要留空格,防止误判
priority_queue<int, vector<int>, greater<int> > qqq;//  定义小的先出队列

q.empty()     //  如果队列为空,则返回true,否则返回false
q.size()      //  返回队列中元素的个数
q.pop()       //  删除队首元素,但不返回其值
q.top()       //  返回具有最高优先级的元素值,但不删除该元素
q.push(item)  //  在基于优先级的适当位置插入新元素

//deque双端队列
#include<deque>
deque<int> dep;

deq.push_front(const T& x); //头插
deq.push_back(const T& x);	//尾插
deq.insert(iterator it, const T& x);	//任意位置

deq.pop_front();	//删除头部
deq.pop_back();		//删除尾部
deq.erase(iterator it);		//删除任意

deq[1]; // 并不会检查是否越界
deq.at(1); // 以上两者的区别就是 at 会检查是否越界,是则抛出 out of range 异常
deq.front();	//访问头部
deq.back();		//访问尾部
\end{lstlisting}


\subsection{map}
\begin{lstlisting}
map<T1, T2> mp;
mp[key] = value;
T2 value = mp[key];
mp.size();       //  返回元素个数
mp.empty();      //  判断是否为空
mp.clear();      //  清空所有元素
\end{lstlisting}


\subsection{bitset}
\begin{lstlisting}
const int MAXN = 32;

bitset<MAXN> bt;            //  bt 包括 MAXN 位,下标 0 ~ MAXN - 1,默认初始化为 0
bitset<MAXN> bt1(0xf);      //  0xf 表示十六进制数 f,对应二进制 1111,将 bt1 低 4 位初始化为 1
bitset<MAXN> bt2(012);      //  012 表示八进制数 12,对应二进制 1010,即将 bt2 低 4 位初始化为 1010
bitset<MAXN> bt3("1010");   //  将 bt3 低 4 位初始化为 1010

bt.any()        //  bt 中是否存在置为 1 的二进制位?
bt.none()       //  bt 中不存在置为 1 的二进制位吗?
bt.count()      //  bt 中置为 1 的二进制位的个数
bt.size()       //  bt 中二进制位的个数
bt[pos]         //  访问 bt 中在 pos 处的二进制位
bt.test(pos)    //  bt 中在 pos 处的二进制位是否为 1
bt.set()        //  把 bt 中所有二进制位都置为 1
bt.set(pos)     //  把 bt 中在 pos 处的二进制位置为 1
bt.reset()      //  把 bt 中所有二进制位都置为 0
bt.reset(pos)   //  把 bt 中在pos处的二进制位置为0
bt.flip()       //  把 bt 中所有二进制位逐位取反
bt.flip(pos)    //  把 bt 中在 pos 处的二进制位取反
bt[pos].flip()  //  同上
bt.to_ulong()   //  用 bt 中同样的二进制位返回一个 unsigned long 值
\end{lstlisting}


\subsection{algorithm}
\begin{lstlisting}
reverse(begin, end)	// 反转

unique(begin, end)	// 需排序,去除重复的相邻元素,常用于求不同元素个数
int n=unique(a, a+10)-a;

lower_bound(begin, end, value)	//返回指向第一个不小于给定值的元素的迭代器

upper_bound(begin, end, value)	//返回指向第一个大于给定值的元素的迭代器

next_permutation(array)	//一种排列的下一种排列
\end{lstlisting}



\section{计算几何}
\subsection{三角形面积}
\begin{lstlisting}
\\ 海伦公式
int p = (a+b+c)/2;
int s = sqrt(p * (p - a) * (p - b) * (p - c));

\\ 两边和夹角
\\ a, b为边, x为a和b的夹角
s = 0.5 * a * b * sin(x/90.0*acos(0));
\end{lstlisting}


\section{其它}
\subsection{数据类型范围}
\begin{tabular}{|c|c|}
	\hline 数据类型&范围\\
	\hline char&-128 - 127\\
	\hline int&2147483648 - 2147483647(十位)\\
	\hline long long&-9223372036854775808 - 9223372036854775807(大约十九位)\\
	\hline double&1.7 * 10\^308\\
	\hline
\end{tabular}

\subsection{头文件}
\begin{lstlisting}
//#include<stdio.h>
//#include<iostream>
//#include<queue>
//#include<algorithm>
//#include<cstring>
//#include<vector>
//#include<cmath>
//#include<string>
//#include<map>
//#include<set>
#include<bits/stdc++.h>
using namespace std;
typedef long long ll;
#define inf 0x3f3f3f3f
typedef pair<int, int> P;
const int maxn = 5e4+5;
const ll mod = 1e9+7;
\end{lstlisting}

\subsection{Vim配置}
\begin{lstlisting}
// open ~/.vimrc
syntax on
set nu ts=4 sw=4 mouse=a cin
colo desert

map <C-A> ggVG"+y
map <F5> :call CR()<CR>
func! CR()
exec "w"
exec "!g++ -O2 -g -std=c++11 -Wall % -o %<"
exec "! ./%<"   
endfunc
\end{lstlisting}

\subsection{输入挂}
\subsubsection{关闭同步}
\begin{lstlisting}
#define endl '\n'

ios::sync_whit_stdio(0);
cin.tie(0);
\end{lstlisting}

\subsubsection{IO}
\begin{lstlisting}
#include<cstdio>

inline void read(int &x) 	//看情况可去掉负数部分
{
	int t = 1;
	char ch = getchar();
	while(ch < '0' || ch > '9') { if(ch == '-') t = -1; ch = getchar();}
	x = 0;
	while(ch >= '0' && ch <= '9'){ x=x*10+ch-'0'; ch = getchar(); } 
	x*=t;
}

void print(int i){
	if(i<10){
		putchar('0'+i);
		return ;
	}
	print(i/10);
	putchar('0'+i%10);
}
\end{lstlisting}

\subsection{C++大数}
\subsubsection{大数加法}
\begin{lstlisting}
string add(string a,string b)
{	//两数相加
	string res="";
	int i=1;
	string first="0";
	while(true) {
		int tai=a.size()-i;
		int tbi=b.size()-i;
		if(tai<0 && tbi<0)
			break;			//从两数最右边开始模拟加法运算直到两数都遍历完
		int ta,tb;
		if(tai<0)
			ta=0;	//如果没数则至为0
		else
			ta=a[tai]-'0';
		if(tbi<0)
			tb=0;	//如果没数则至为0
		else
			tb=b[tbi]-'0';
		int temp=ta+tb+first[0]-'0';	//相加 first保存上一个的进位信息
		first[0]=temp%10+'0';	//当前位是对10取余
		res=first+res;
		first[0]=temp/10+'0';	//进位是除10
		i++;
	}
	if(first!="0")
		res=first+res;	//如果进位还有则添加
	if(res[0]=='0' && res.size()>1)	//去除前导0
		res.erase(res.begin());
	return res;
}
\end{lstlisting}

\subsubsection{大数乘法}
\begin{lstlisting}
// 还需要配合大数加法
string mul(string a,string b) {
	if(b.size()==1) {	//如果b只有一位则使其分别相乘
		string res="";
		string first="0";
		int mb=b[0]-'0';
		for(int i=a.size()-1; i>=0; i--) {	//从最后开始依次计算
			int temp=(a[i]-'0') * mb + (first[0]-'0');
			first[0]=temp%10+'0';	//当前位
			res=first+res;
			first[0]=temp/10+'0';	//进位
		}
		if(first!="0") {
			res=first+res;	//处理进位
		}
		if(res[0]=='0' && res.size()>1)
			res.erase(res.begin());	//除去前导0
		return res;
		}
	//否则则把b拆分为一位
	string res="0";
	string zero="";
	for(int i=b.size()-1; i>=0; i--) {	//从b的最后一位开始
		string temp=mul(a,b.substr(i,1));	//计算当前为与a相乘
		res=add(res,temp+zero);	//在其后添加适当的0再与结果相加
		zero=zero+"0";
	}
	return res;
}
\end{lstlisting}

\subsubsection{整数转string}
\begin{lstlisting}
string inttostring(int m) {	//将整数转为string
	string res="";
	string temp="0";
	while(m) {
		temp[0]=m%10+'0';
		res=temp+res;
		m/=10;
	}
	if(res=="")
		res="0";
	return res;
}
\end{lstlisting}


\subsection{Java}
\begin{lstlisting}
valueOf(parament); 将参数转换为制定的类型

比如 int a=3;

BigInteger b=BigInteger.valueOf(a);

则b=3;

String s=”12345”;

BigInteger c=BigInteger.valueOf(s);

则c=12345;

// 常用函数
1.赋值:
BigInteger a=new BigInteger("1");
BigInteger b=BigInteger.valueOf(1);

2.运算:
① add(); 大整数相加 
BigInteger a=new BigInteger(“23”); 
BigInteger b=new BigInteger(“34”); 
a.add(b);

②subtract(); 相减 
③multiply(); 相乘 
④divide(); 相除取整 
⑤remainder(); 取余 
⑥pow(); a.pow(b)=a^b 
⑦gcd(); 最大公约数 
⑧abs(); 绝对值 
⑨negate(); 取反数 
⑩mod(); a.mod(b)=a%b=a.remainder(b); 

3.BigInteger构造函数: 
一般用到以下两种: 
BigInteger(String val); 
将指定字符串转换为十进制表示形式; 
BigInteger(String val,int radix); 
将指定基数的 BigInteger 的字符串表示形式转换为 BigInteger

4.基本常量: 
A=BigInteger.ONE 1 
B=BigInteger.TEN 10 
C=BigInteger.ZERO 0 

5.n.compareTo(BigInteger.ZERO)==0  //相当于n==0

6.if(a[i].compareTo(n)>=0 && a[i].compareTo(m)<=0)   // a[i]>=n && a[i]<=m

// 模板
import java.math.BigInteger;
import java.math.BigDecimal;
import java.util.Scanner;
import java.util.*;
import java.io.*;
public class Main {
	public static void main(String [] args){
		Scanner cin = new Scanner(System.in);
		BigInteger a, b;
		while(cin.hasNext())//相当于c语言中的scanf("%d", &n) != EOF
		{
			a = cin.nextBigInteger();
			b = cin.nextBigInteger();
			System.out.println(a.add(b));//大整数加法
			System.out.println(a.subtract(b));//大整数减法
			System.out.println(a.multiply(b));//大整数乘法
			System.out.println(a.divide(b));//大整数除法,取整
			System.out.println(a.remainder(b));//大整数取模
			System.out.println(a.abs());//对大整数a取绝对值
			int x = 0;
			System.out.println(a.pow(x));//大整数a的x次幂
			int y = 8;
			System.out.println(a.toString(y));//返回大整数a的p进制用字符串表现的形式
			System.out.println(a.toString());//返回大整数a的十进制用字符串表现的形式
			//大整数之间的比较
			if( a.compareTo(b) == 0 ) System.out.println("a == b"); //大整数a==b
			else if( a.compareTo(b) > 0 ) System.out.println("a > b"); //大整数a>b
			else if( a.compareTo(b) < 0 ) System.out.println("a < b"); //大整数a<b)
			
			BigDecimal c, d;
			c = cin.nextBigDecimal();
			d = cin.nextBigDecimal();
			System.out.println(c.add(d));//浮点数相加
			System.out.println(c.subtract(d));//浮点数相减
			System.out.println(c.multiply(d));//浮点数相乘
		}
	}
}

// 输入方式
int a = cin.nextInt();
String s = cin.nextLine();
double b = cin.nextDouble();
\end{lstlisting}